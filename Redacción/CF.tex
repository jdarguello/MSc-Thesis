\subsection{Metodolog\'ia de frontera} \label{CondF}

\noindent
\justify

La mezcla s\'olido-l\'iquida en la entrada presenta un caudal de $0.3 \left[m^3 /h \right]$. La temperatura es constante durante todo el proceso e igual a $28 [\degree C]$. En el tiempo $t=0$, las part\'iculas s\'olidas esf\'ericas se generan en una regi\'on lineal, como se aprecia en la Figura \ref{generation}.

\begin{figure}[h!]
	\begin{tikzpicture}
		\draw (0,0) -- (3,0);
		\draw (3,1) -- (0,1);
		\draw plot [smooth, tension=1] coordinates { (3,0) (3.1,0.25) (3,0.5) (2.9,0.75) (3,1) (3.1,0.75) (3,0.5) };
		
		\draw[red!10!blue, arrows={-Triangle[angle=90:3pt,red!10!blue,fill=red!10!blue]}] (-1,0.5) -- (1,0.5) node [black, text width=2mm] {$\dot{m} _{sol}$};
		
		\draw [dashed, red!50!black] (2.2, -0.2) -- (2.2, 1.2);
		
		\draw[red!30!blue, arrows={-Triangle[angle=90:3pt,red!30!blue,fill=red!30!blue]}] (2.6,0.5) -- (3.4,0.5) node [black, text width=0.5cm] {$\dot{m} _{sol}+\dot{m} _{part}$};
	\end{tikzpicture}
	\caption{Zona de generaci\'on de part\'iculas en la tuber\'ia de entrada.}
	\label{generation}
\end{figure}

\noindent
\justify

De la Figura \ref{generation}: $\dot{m} _{sol}$ se refiere al flujo m\'asico de solvente, $\dot{m} _{part}$ es el flujo m\'asico de part\'iculas generadas y las l\'ineas punteadas representan la zona de generaci\'on.

\noindent
\justify

Las condiciones de frontera se resumen en la Figura \ref{ConFron}.

%Datos geométricos
\def\D{2.7}    	%Diámetro de ingreso
\def\H{3}		%Altura de la zona de lodos
\def\A{2.5}		%Ancho del panel de lamelas
\def\HL{3}		%Altura del panel de lamelas
\def\nL{10}		%Número de lamelas
\def\ang{60}	%Ángulo de inclinación del panel

\def\xlam{\x+\A+0.5*\D}

\def\fin{\nL-1}

%Condiciones de frontera
\def\V{4.44}
\def\P{0}

%Referencia
\def\x{-\A/2-0.5/2*\D-\HL/2}
\def\y{0}

\begin{figure}[h!]
	\centering
	\begin{adjustbox}{max width = 0.5\textwidth}
	\begin{tikzpicture}
		%Geometría general
		\draw (\x,\y) -- (\x,-\D) -- (\x+0.5*\D,-\D) -- 
			(\x+0.5*\D, -2*\D) -- (\x+\A+0.5*\D, -2*\D) -- 
			(\x+\A+0.5*\D, \y) -- ({\xlam + \HL*cos(\ang)},{\y + \HL*sin(\ang)}) -- ({\xlam + \HL*cos(\ang) - \A/\nL}, {\y + \HL*sin(\ang)}) -- (\xlam - \A/\nL, \y) -- cycle;
		
		%Condición entrada
		\draw (\x - 1, \y) -- node[above, rotate=90]  {$\vec{V_0}$} (\x - 1, \y - \D);
		\foreach \yL in {0, ..., 10}
			\draw[red!10!black, arrows={-Triangle[angle=90:3pt,red!10!black,fill=red!10!black]}] (\x-1,\y - \yL*\D/10) -- (\x-0.05,\y - \yL*\D/10);
		
		\draw[red!10!black, arrows={-Triangle[angle=90:3pt,red!10!black,fill=red!10!black]}] (\x,{\y + \HL*sin(\ang)}) -- node[right] {$\vec{g}$} (\x, \y + \H/2);
		%Condición salida		
		\draw ({\xlam + \HL*cos(\ang) - \A/\nL/2}, {\y + \HL*sin(\ang)}) -- ({\xlam + \HL*cos(\ang) - \A/\nL},{\y + \HL*sin(\ang) + \A/10}) -- ({\xlam + \HL*cos(\ang)},{\y + \HL*sin(\ang) + \A/10}) node[above] {$P_0 = P_{atm}$} -- cycle;
	\end{tikzpicture}
	\end{adjustbox}
	\caption{Condiciones de frontera.}
	\label{ConFron}
\end{figure}

\begin{table}[h!]
	\centering
	\begin{tabular}{|c|c|c|}
		\hline
		\textbf{Zona} & \textbf{Propiedad}  & \textbf{Tipo} \\ \hline
		\textit{Entrada} & Velocidad $(V_0)$ & Neumann \\ \hline
		\textit{Salida} & Presi\'on $(P_0)$ & Dirichlet \\ \hline
	\end{tabular}
	\caption{Clasificaci\'on de las condiciones de frontera.}
	\label{CFT}
\end{table}

\noindent
\justify

La clasificaci\'on de las condiciones de frontera se puede apreciar en el Cuadro \ref{CFT}. Al tratarse de una simulaci\'on 2D, las caras frontal y posterior de la geometr\'ia se consideran superficies vac\'ias.
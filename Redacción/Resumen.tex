\begin{center}
	\textbf{\textsc{{\Large Resumen}}}\\
\end{center}

\noindent
\justify

\begin{table}[h!]
\begin{tabular}{ll}
\textbf{T\'itulo:} & \noindent\parbox{0.6\textwidth}{Dise\~no del sistema de eluci\'on y filtrado de una planta de extracci\'on\footnotemark} \\
 & \\
\textbf{Autor:} & \noindent\parbox{0.6\textwidth}{Juan David Arg\"uello Plata\footnotemark} \\
 & \\
\textbf{Palabras clave:} & \noindent\parbox{0.6\textwidth}{CFD-DEM, MSPD, Python, Jupyter, ParaView, OpenFoam, Yade} \\
\end{tabular}
\end{table}


\noindent
\justify


\textbf{\large Contenido:} 

\noindent
\justify

El proceso de extracci\'on basado en el m\'etodo de dispersi\'on de la matriz en fase s\'olida, MSPD, consiste de tres etapas. La primera es la etapa de pretratamiento, o de molienda, en donde se busca disminuir el tama\~no de part\'icula del material org\'anico con el fin de incrementar el \'area de transferencia de masa. La siguiente se trata de la etapa de eluci\'on y filtrado, en donde se produce la extracci\'on de metabolitos secundarios a trav\'es de un solvente; luego, se filtra el material particulado para obtener la mezcla homog\'enea solvente - extracto. Finalmente, se desarrolla una etapa de separaci\'on de sustancias, en donde se separa el solvente del extracto (producto final).

\noindent
\justify

Se propone una metodolog\'ia de dise\~no autom\'atico del sistema de eluci\'on y filtrado que simula el comportamiento fluidodin\'amico durante la etapa de filtrado, permitiendo predecir el grado de concentraci\'on de part\'iculas a lo largo del sistema a trav\'es de un modelo num\'erico basado en CFD-DEM. Esta metodolog\'ia ha sido elaborada con herramientas de c\'odigo abierto. Utilizando Python como lenguaje base, Jupyter como entorno de desarrollo, ParaView como plataforma de an\'alisis de resultados y librer\'ias de C++ (como Yade, LIGGGHTS y OpenFoam) para el desarrollo de las simulaciones num\'ericas. 

\footnotetext[1]{Tesis de grado de maestr\'ia en ingenier\'ia mec\'anica.}
\footnotetext[2]{\underline{Facultad:} F\'isicomecanicas. \underline{Escuela:} Ingenier\'ia mec\'anica. \underline{Director:} Omar Armando G\'elvez Arocha.}
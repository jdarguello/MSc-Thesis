\begin{center}
	\section{Modelo CFD-DEM}
\end{center}

\subsection{Selecci\'on de par\'ametros}

\noindent
\justify

En las simulaciones CFD-DEM, se inicia definiendo las condiciones de frontera: la mezcla s\'olido-l\'iquida en la entrada presenta un caudal de $0.3 \left[m^3 /h \right]$. Los muros se consideran como superficies no deslizantes para la fase l\'iquida. La temperatura es constante durante todo el proceso e igual a $27 [\degree C]$. Se trata de una simulaci\'on 2D en donde se emplea una malla CFD tetrah\'edrica de X nodos y X elementos. En el tiempo $t=0$, las part\'iculas s\'olidas esf\'ericas se generan en una regi\'on lineal de base $0.01 [m]$ de longitud, como se aprecia en la Figura \ref{generation}.

\begin{figure}[h!]
	\begin{tikzpicture}
		\draw (0,0) -- (3,0);
		\draw (3,1) -- (0,1);
		\draw plot [smooth, tension=1] coordinates { (3,0) (3.1,0.25) (3,0.5) (2.9,0.75) (3,1) (3.1,0.75) (3,0.5) };
		
		\draw[red!10!blue, arrows={-Triangle[angle=90:3pt,red!10!blue,fill=red!10!blue]}] (-1,0.5) -- (1,0.5) node [black, text width=2mm] {$\dot{m} _{sol}$};
		
		\draw [dashed, red!50!black] (2.2, -0.2) -- (2.2, 1.2);
		
		\draw[red!30!blue, arrows={-Triangle[angle=90:3pt,red!30!blue,fill=red!30!blue]}] (2.6,0.5) -- (3.4,0.5) node [black, text width=0.5cm] {$\dot{m} _{sol}+\dot{m} _{part}$};
	\end{tikzpicture}
	\caption{Zona de generaci\'on de part\'iculas en la tuber\'ia de entrada.}
	\label{generation}
\end{figure}
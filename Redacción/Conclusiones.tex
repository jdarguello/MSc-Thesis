\begin{center}
	\section{Conclusiones}
\end{center}


\begin{enumerate}
	\item La presencia de v\'ortices y remolinos en la simulaci\'on CFD-DEM rectifica la decisi\'on de haber empleado un solucionador basado en \texttt{pimpleFoam}, el cual se adapt\'o para resolver el problema \textit{Euler - Lagrange} (E - L).
	\item Con base en los resultados obtenidos en la secci\'on \ref{CFDEM:resultados}, y analizados en la secci\'on \ref{CFDEM:analisis}, se reafirma la importancia de las simulaciones num\'ericas basadas en CFD-DEM para el dise\~no de maquinaria y equipos que busquen separar mezclas de sustancias s\'olidas y l\'iquidas. De la Figura \ref{CFDEM:part} se concluye que el panel de lamelas presenta zonas de salidas \textit{limpias} de material particulado y otras con una concentraci\'on menor a la inicialmente procesada. Futuras investigaciones podr\'ian emplear el modelo CFD-DEM desarrollado en el presente trabajo como base para el dise\~no de un nuevo sistema de sedimentaci\'on que emplee un \textit{reflujo} de las zonas con posibles impurezas para garantizar una completa separaci\'on de sustancias.
	\item Se implement\'o la metodolog\'ia del modelo CFD-DEM con base en herramientas de c\'odigo abierto; demostrando la viabilidad de estas herramientas en el desarrollo de nuevos productos y servicios de bajos costos de inversi\'on inicial y alto grado de innovaci\'on.
	\item De acuerdo al proceso de validaci\'on desarrollado en el cap\'itulo \ref{validacion}, se concluye que el modelo CFD-DEM desarrollado es preciso y puede ser empleado en diversas aplicaciones de flujo, bien sea laminar o turbulento, en donde part\'iculas s\'olidas interact\'uan con fluidos newtonianos en fase l\'iquida o gaseosa.
\end{enumerate}

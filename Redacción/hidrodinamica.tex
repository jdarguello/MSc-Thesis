\paragraph{An\'alisis hidrodin\'amico de una part\'icula} \label{hidroD}

\noindent
\justify

La \textit{sedimentaci\'on discreta} se refiere a un modelo de sedimentaci\'on en donde las part\'iculas s\'olidas no tienden a aglomerarse ni a colisionar entre s\'i. El comportamiento de los s\'olidos se encuentra, \'unicamente, en funci\'on de las propiedades de las part\'iculas y del fluido directamente. El balance de fuerzas sobre una part\'icula se puede apreciar directamente en la Figura \ref{partF}.

\begin{figure}[h!]
	\centering
	\begin{tikzpicture}
		\draw[pattern = dots, pattern color= brown] (0,0) circle (2);
		\node[align=center] at (0,0) {Spherical\\ particle};
		\draw[-triangle 45, gray!30!black] (0,3) node [above] {$\vec{W}$} -- (0,0.5) ;
		\foreach \x in {0, ..., 10}
			\draw[-triangle 45, blue!70!white] (0.4*\x-2,{-sqrt(2^2-(0.4*\x-2)^2)-0.5}) -- (0.4*\x-2,{-sqrt(2^2-(0.4*\x-2)^2)});
		\foreach \x in {0, ..., 10}
			\draw[-triangle 45, blue!70!white] (0.4*\x-2,{sqrt(2^2-(0.4*\x-2)^2)+0.5}) -- (0.4*\x-2,{sqrt(2^2-(0.4*\x-2)^2)});
		\draw[-triangle 45, gray!80!white] (0,-3) node [below] {$\vec{F}$} -- (0,-2);
		\draw[-triangle 45, dashed] (-3,2) -- (-3,0) node [below] {$\vec{U}$};
		\draw[-triangle 45, dashed] (3,2) -- (3,0) node [below] {$\vec{a}$};
	\end{tikzpicture}
	\caption{Din\'amica de la part\'icula.}
	\label{partF}
\end{figure}

\noindent
\justify


De la Figura \ref{partF}, $\vec{W}$ se refiere al peso de la part\'icula, $\vec{F}$ a la fuerza de arrastre, $\vec{U}$ es la velocidad de asentamiento y $\vec{a}$ es la aceleraci\'on de asentamiento.

\noindent
\justify

Romero$^{\cite{RobertoRojas}}$ indica que la fuerza de arrastre se puede calcular con base en la Ecuaci\'on \ref{farr}.

\begin{equation}
	F = \frac{C \, A_n \, \rho _f \, U^2}{2}
	\label{farr}
\end{equation}

\noindent
\justify

De la Ecuaci\'on \ref{farr}, $C$ es el coeficiente de arrastre de Newton, $A_n$ es el \'area transversal de la part\'icula en la direcci\'on de asentamiento, $U$ es la velocidad de asentamiento y $\rho _f$ es la densidad del fluido.

\noindent
\justify

El peso de la part\'icula en el fluido depende directamente de la gravedad y de las densidades del fluido y de la part\'icula, como se aprecia en la ecuaci\'on \ref{pesoP}.

\begin{equation}
	W = V \left(\rho _s - \rho _f \right) g
	\label{pesoP}
\end{equation}

\noindent
\justify

D\'onde: $V$ es el volumen de la part\'icula, $\rho _s$ es la densidad de la part\'icula, $\rho _f$ es la densidad del fluido y $g$ corresponde a la aceleraci\'on de la gravedad.

\noindent
\justify

El coeficiente de arrastre es funci\'on del n\'umero de Reynolds:

\begin{equation}
	Re _s = \frac{d_p \, U}{\mu}
	\label{Res}
\end{equation}

\noindent
\justify

D\'onde $d_p$ es el di\'ametro de la part\'icula y $\mu$ la viscosidad cinem\'atica del fluido. Se estipula que para part\'iculas esf\'ericas y $Re _s < 10000$ el coeficiente de arrastre se puede calcular de la siguiente forma:

\begin{equation}
	C = \frac{24}{Re_s} + \frac{3}{Re _s ^{1/2}} + 0.34
	\label{CoefArr}
\end{equation}

\noindent
\justify

En un principio, se espera que en el decenso la part\'icula acelere hasta que la fuerza de arrastre sea igual a la fuerza impulsora del asentamiento. Cuando las fuerzas verticuales se encuentran en equilibrio, la velocidad ser\'a constante. De esta manera, es posible relacionar las Ecuaciones \ref{pesoP} y \ref{farr}:

\begin{equation}
	U_{max} = \sqrt{\frac{2 V \left(\rho _s - \rho _f \right) g}{C \, A_n \, \rho _f}}
	\label{Uas}
\end{equation}

\noindent
\justify

La Ecuaci\'on \ref{Uas} se conoce como la Ley de Stokes y ha sido comprobada de manera experimental$^{\cite{RobertoRojas}}$. 
\subsubsection{CFD-DEM}

\noindent
\justify

El acoplamiento entre CFD \textit{(``Computational Fluid Dynamics")} y DEM \textit{(``Discrete Element Method")} busca predecir la interacci\'on fluido-part\'icula. El flujo se resuelve a trav\'es de CFD basado en malla (ver secci\'on \ref{CFD:problema}), mientras que la fase s\'olida es modelada mediante DEM para cada part\'icula sujeta a trav\'es de fuerzas hidrodin\'amicas, fuerzas de cuerpo (como la gravedad) y a trav\'es de fuerzas de contacto; actualizando valores de velocidad y posici\'on conforme a la segunda ley de Newton (Hoomans \textit{et al.}, 1996; Tsuji \textit{et al.}, 1993; Xu y Yu, 1997). 

\noindent
\justify

La tasa de part\'iculas debido a la sedimentaci\'on, a trav\'es de canales inclinados, ha sido ampliamente estudiado debido al reconocido \textit{efecto Boycott}$^{\cite{Boycott}}$. Este fen\'omeno se produce por el incremento en el \'area efectiva de sedimentaci\'on debido a la presencia de placas inclinadas$^{\cite{Boycott2}}$. Acrivos \textit{et al.} han desarrollado una serie de planteamientos te\'oricos y experimentales entre la tasa de sedimentaci\'on de part\'iculas y el \'area efectiva de sedimentaci\'on. El efecto Boycott ha sido aplicado con \'exito en diversos procesos industriales para la remoci\'on de part\'iculas en lecho de fluidizado a trav\'es del asentamiento gravitacional; entre los principales ejemplos de este hecho se encuentran: tratamientos de aguas residuales$^{\cite{aguares}}$ y procesos de filtrado de agua$^{\cite{aguafil}}$.

\noindent
\justify

El acercamiento experimental para la investigaci\'on caracter\'istica de part\'iculas en suspenci\'on a alta concentraci\'on ha demostrado ser una experiencia retadora debido a las limitantes instrumentales y t\'ecnicas de medici\'on$^{\cite{articulo}}$. Los modelos num\'ericos basados en CFD han demostrado ser una herramienta poderosa y promisoria que provee informaci\'on detallada y precisa sobre las caracter\'isticas locales del flujo particulado. Normalmente, se han aplicado dos enfoques generales en la literatura para resolver problemas que involucran flujo particulado: \textit{Eulerian - Eulerian} (E-E) y \textit{Eulerian - Lagrange} (E-L). En el enfoque E-E, las fases s\'olida y el fluida son interpretadas de manera continua en donde comparten el mismo gripo de ecuaciones gobernantes. Doroodchi \textit{et al.}$^{\cite{ref1}}$ emplearon el modelo E-E para investigar la influencia de las placas inclinadas y el efecto expansivo de s\'olidos en suspensi\'on en camas de lecho fluidizado; obteniendo resultados prometedores tanto en la parte experimental como num\'erica. Salem \textit{et al} $^{\cite{ref2}}$ desarrollaron un modelamiento en CFD empleando el modelo E-E para evaluar las caracter\'isticas hidr\'aulicas de un sedimentador de placas hidr\'aulicas (IPS, por sus siglas en ingl\'es); demostrando que el importante rol que cumplen las herramientas computacionales en el estudio de los sistemas de sedimentaci\'on. Sin embargo, el tratamiento de la fase s\'olida de manera continua va en contra de la naturaleza discreta de las part\'iculas s\'olidas, y todav\'ia m\'as importante: el acercamiento a trav\'es de E-E carece de facultades num\'ericas para revelar informaci\'on importante referente a la escala particular.

\noindent
\justify

El enfoque otorgado por el modelo E-L, por otro lado, adopta la teor\'ia continua para la fase l\'iquida y resuelve el problema cinem\'atico de cada part\'icula individual \textbf{directamente}. Informaci\'on 
\begin{center}
	\section{Introducci\'on}
\end{center}

\noindent
\justify

El m\'etodo de \textit{dispersi\'on de la matriz en fase s\'olida}, MSPD, es un m\'etodo de extracci\'on ampliamente usado a escala de laboratorio para la obtenci\'on y an\'alisis de la actividad biol\'ogica de extractos. Consiste de tres etapas: pretratamiento eluci\'on y filtrado, y separaci\'on de sustancias. El \'exito de este m\'etodo extractivo recae en su simplicidad, rapidez y econom\'ia$^{\cite{barker2007}}$; razones por las que se han desarrollado estudios de escalabilidad en busca de la industrializaci\'on$^{\cite{Proyecto, Stashenko2017, Patente2018}}$. La presente investigaci\'on desarrolla una metodolog\'ia de dise\~no enfocada en la etapa de eluci\'on y filtrado de una planta de extracci\'on, basada en el m\'etodo MSPD. Esta metodolog\'ia de dise\~no emplea m\'etodos num\'ericos que garantizan el correcto desempe\~no y funcionalidad del sistema.

\noindent
\justify

Los m\'etodos num\'ericos son teoremas matem\'aticos que permiten describir la naturaleza de diferentes fen\'omenos de car\'acter f\'isico-qu\'imico. Son ampliamente usados en ingenier\'ia como metodolog\'ias predictivas durante el proceso de dise\~no funcional y mec\'anico. Para el an\'alisis de comportamientos fluidodin\'amicos de part\'iculas, es com\'un encontrar estudios que combinen los m\'etodos num\'ericos de \textit{elementos discretos} y \textit{vol\'umenes finitos}, o como es mejor conocido: \textit{m\'etodo CFD-DEM}.

\noindent
\justify

El acoplamiento entre CFD-DEM es empleado cuando se busca desarrollar an\'alisis de part\'iculas y su interacci\'on en medios viscosos. Ampliamente usado para an\'alisis de lecho fluidizado$^{\cite{Alobaid2013}}$, separadores de cicl\'on$^{\cite{Chu2011}}$ y para el estudio de retenci\'on de part\'iculas en medios filtrantes$^{\cite{Yue2016}}$, por citar algunos ejemplos. Se han desarrollado estudios experimentales que corroboran la efectividad y viabilidad de las simulaciones num\'ericas que emplean CFD-DEM$^{\cite{Alobaid2013, Varas2017}}$.

\noindent
\justify

La metodolog\'ia de dise\~no del sistema de eluci\'on y filtrado de la planta de extracci\'on, basada en el m\'etodo CFD-DEM, permite analizar la interacci\'on fluido-part\'icula a trav\'es de diferentes resultados; entre ellos: perfiles de velocidad, de presi\'on y grado de concentraci\'on de part\'iculas a lo largo del volumen de control. Esta metodolog\'ia se elabor\'o con herramientas de \textit{c\'odigo abierto}; empleando Python como lenguaje base, Jupyter como entorno de desarrollo, ParaView como plataforma de an\'alisis de resultados y librer\'ias de C++ (entre ellas: LIGGGHTS y OpenFOAM) para el desarrollo de las simulaciones num\'ericas. El modelo num\'erico desarrollado ha sido validado a trav\'es de la comparaci\'on directa entre los resultados obtenidos por \'este con un caso reportado en la literatura.

\newpage

\begin{center}
	\section{Planta de extracci\'on}
\end{center}

\noindent
\justify

El concepto de \textit{extracci\'on} se concibe como la obtenci\'on de un producto ``A", procedente de una materia prima ``B", mediante procesos f\'isico-qu\'imicos de separaci\'on de sustancias. La naturaleza de esta materia prima, procesada en la planta de extracci\'on de estudio, es de origen org\'anico; trat\'andose de hojas, ramas, frutos, flores y ra\'ices procedentes de \textbf{plantas arom\'aticas y medicinales}.

%Función de hoja
\newcommand{\hoja}[4]%Cantidad de argumentos
{
	%Contorno
	\draw[pattern=north west lines, pattern color=green!55!black] plot [smooth cycle] coordinates{(#1-#3, #2) (#1 + 0.2*#3, #2 -0.2*#3) (#1 + 2.2*#3, #2 + 1.2*#3) (#1 + 2.9*#3, #2 + 2.7*#3) (#1 + 2.8*#3, #2 + 1.8*#4) (#1 + 2.5*#3, #2 + 2*#4) (#1 + 2.48*#3, #2 + 2.2*#4) (#1 + 2.2*#3, #2 + 2.08*#4) (#1 + 0.7*#3, #2 + 1.9*#4) (#1 - 0.1	*#3, #2 + 3*#3) (#1 - 0.9*#3, #2 + 1.4*#3)};
	\draw[white] plot [smooth cycle] coordinates{(#1-#3, #2) (#1 + 0.2*#3, #2 -0.2*#3) (#1 + 2.2*#3, #2 + 1.2*#3) (#1 + 2.9*#3, #2 + 2.7*#3) (#1 + 2.8*#3, #2 + 1.8*#4) (#1 + 2.5*#3, #2 + 2*#4) (#1 + 2.48*#3, #2 + 2.2*#4) (#1 + 2.2*#3, #2 + 2.08*#4) (#1 + 0.7*#3, #2 + 1.9*#4) (#1 - 0.1	*#3, #2 + 3*#3) (#1 - 0.9*#3, #2 + 1.4*#3)};;
	%Raíz interna
	\draw[white, fill=green!10!white] (#1-1.9*#3, #2-1.5*#3) -- (#1-1.95*#3, #2-1.3*#3) -- (#1-0.2*#3, #2+0.8*#3) -- (#1+0.9*#3, #2+1*#3) -- (#1-0.15*#3, #2+0.9*#3) -- (#1+1.1*#3, #2+2.3*#3) -- (#1+2.1*#3, #2+2.45*#3) -- (#1+1.12*#3, #2+2.4*#3) -- (#1+1.65*#3, #2+3*#3) -- (#1+2.55*#3, #2+3.05*#3) -- (#1+1.7*#3, #2+3.1*#3) -- (#1+2.25*#3, #2+3.7*#3) -- (#1+1.6*#3, #2+3.1*#3) -- (#1+0.7*#3, #2+3.05*#3) -- (#1+1.55*#3, #2+3*#3) -- (#1+0.97*#3, #2+2.4*#3) -- (#1+0.3*#3, #2+2.45*#3) -- (#1+0.9*#3, #2+2.3*#3) -- (#1-0.3*#3, #2+0.9*#3) -- (#1-0.8*#3, #2+1*#3) -- (#1-0.35*#3, #2+0.8*#3) -- (#1-2.1*#3, #2-1.3*#3) -- cycle;
}

\newcommand{\circuloH}[4]
{
	%Círculo
	\hoja{#1}{#2}{#3}{#4};
	\draw[pattern=north west lines, pattern color=green!55!black] (#1 -0.4*#3, #2 + 3*#3) circle (0.9*#3);
	\draw[white] (#1 -0.4*#3, #2 + 3*#3) circle (0.9*#3);
	%Interior
	\draw[white, fill=green!10!white] (#1 - 0.8*#3, #2 + 2.2*#3) -- (#1 +0.2*#3, #2 + 3.65*#3) arc (45:55:0.9*#3) -- (#1 - 0.98*#3, #2 + 2.3*#3) -- cycle;
	%\draw (#1 - 1.08*#3, #2 + 2.4*#3) -- (#1 -0.1*#3, #2 + 3.83*#3);
}

\newcommand{\gota}[4]
{
	\draw[pattern=crosshatch dots, pattern color=blue!30!black] plot [smooth cycle] coordinates{(#1, #2) (#1 + 0.9*#3, #2 + 0.2*#3) (#1 + #3, #2 + #3) (#1, #4) (#1 - #3, #2 + #3) (#1 - 0.9*#3, #2 + 0.2*#3)};
}

\newcommand{\Gota}[4]
{
	\gota{#1}{#2}{#3}{#4};
	\draw[pattern=dots, pattern color=blue!30!white] plot [smooth cycle] coordinates{(#1, #2+0.25*#4) (#1 + 0.45*#3, #2 + 0.2*#3) (#1 + 0.75*#3, #2 + 0.75*#3) (#1, 0.75*#4) (#1 - 0.75*#3, #2 + 0.75*#3) (#1 - 0.45*#3, #2 + 0.2*#3)};
}

\begin{figure}[h!]
	\begin{tikzpicture}
		\circuloH{-4}{0}{0.75}{1.5};
		\node[green!50!black, align=left] at (-5.8,2) {Materia\\ prima};
		\Gota{4}{-0.5}{1.5}{2.75};
		\node[blue!30!black, align=left] at (5.8,2) {Solvente};
		
		%Moléculas
		\draw[fill=red] (-4.7, 2.6) circle (0.08);
		\draw[fill=red] (-4.2, 2.5) circle (0.08);
		\draw[fill=red] (-3.9, 2.2) circle (0.08);
		\draw[fill=red] (-4.3, 2) circle (0.08);
		
		\draw[-triangle 45, fill=black] (-4.5, 3.2) -- (-4.2, 2.5);
		\node[anchor=north, red!50!white] at (-4.5, 3.7) {Producto};
		
		%Flecha solvente
		\coordinate (O) at (4,1);
  		\coordinate (A) at (1,-0.8);

  		\draw (O) to [bend right=15] (A);
  		\draw[-triangle 45, fill=black] (1.01,-0.79) -- (1,-0.8);
  		
  		%Flecha hoja
		\coordinate (OO) at (-3.5,1.5);
  		\coordinate (AA) at (-1,-0.8);

  		\draw (OO) to [bend left=15] (AA);
  		\draw[-triangle 45, fill=black] (-1.01,-0.79) -- (-1,-0.8);

		
		%---Mezcla---
		%vaso
		\draw (-2,-1) -- (-2,-5) -- (-1,-6) -- (1,-6) -- (2,-5) -- (2, -1) -- cycle;
		\node[align=left] at (3.7,-3.2) {Transferencia\\ de masa};
		%Líquido
		\draw[pattern=crosshatch dots, pattern color=blue!30!black] (-1,-6) -- (1,-6) -- (2,-5) -- (2,-3) -- (-2, -3) -- (-2,-5) -- cycle;		
		%Hoja
		\circuloH{-1}{-4.5}{0.25}{0.5}
		\draw[fill=red] (-1.2, -3.7) circle (0.06);
		\draw[fill=red] (-1, -3.75) circle (0.06);
		\draw[fill=red] (-0.7, -4.3) circle (0.06);
		
		\draw[fill=red] (1.2, -4.6) circle (0.06);
		\draw[fill=red] (1.2, -4) circle (0.06);
		
		\draw[white!40!black, -triangle 45, fill=white] (-1, -3.75) -- (1.2, -4);
		\draw[white!40!black, -triangle 45, fill=white] (-0.7, -4.3) -- (1.2, -4.6);
		
		%---Salidas---
		\hoja{-4}{-11}{0.75}{1.5}
		\node[green!50!black, align=left] at (-5.8,-9.5) {Residuo\\ vegetal};
		
		%Flechas
		\coordinate (OOO) at (-0.8,-6.2);
  		\coordinate (AAA) at (-2.1,-7.5);

  		\draw (OOO) to [bend right=15] (AAA);
  		\draw[-triangle 45, fill=black] (-2,-7.3) -- (-2.1,-7.5);
  		
  		\coordinate (OOOO) at (0.8,-6.2);
  		\coordinate (AAAA) at (2.1,-7.5);

  		\draw (OOOO) to [bend left=15] (AAAA);
  		\draw[-triangle 45, fill=black] (2,-7.3) -- (2.1,-7.5);
		
		%Extracto
		\draw[pattern=crosshatch dots, pattern color=blue!30!black] (2,-8) -- (5,-8) -- (5,-11) -- (2,-11) -- cycle;
		\draw[pattern=dots, pattern color=blue!30!white] (2,-8) -- (5,-8) -- (5,-11) -- (2,-11) -- cycle;
		
		\draw[fill=red] (2.5, -8.5) circle (0.08);
		\draw[fill=red] (2.9, -10) circle (0.08);
		\draw[fill=red] (3.5, -9) circle (0.08);
		\draw[fill=red] (4, -10.5) circle (0.08);
		\draw[fill=red] (4.5, -9.5) circle (0.08);
		
		\node[align=left, blue!30!white!50!black] at (6.4,-9.3) {Extracto o\\ aceite esencial};
		
		
	\end{tikzpicture}
	\caption{Extracci\'on.}
	\label{plant_extraction}
\end{figure}

\newpage

\noindent
\justify

El producto obtenido en la extracci\'on puede tratarse de aceite esencial o extracto, dependiendo de las \textit{condiciones termodin\'amicas} del proceso. El aceite esencial se compone de mol\'eculas de bajo y mediano peso molecular y son empleados por las plantas para garantizar su supervivencia. Pertenecen a diferentes clases de sustancias qu\'imicas (fenoles y terpenos, principalmente)\footnote{Stashenko, E. E. \textit{et al}. Aceites esenciales. Primera edici\'on. Universidad Industrial de Santander. pp. 13 - 20.}. Se carcaterizan por un olor t\'ipico y una alta volatilidad. En t\'erminos productivos, se requiere garantizar condiciones de ebullici\'on durante la etapa de transferencia de masa para su obtenci\'on: el solvente (normalmente agua) que est\'a en contacto con la materia prima puede estar en estado l\'iquido (hidrodestilaci\'on) o gaseoso (destilaci\'on por arrastre con vapor). La mezcla gaseosa entre el vapor del solvente y el aceite esencial es condensada y separada a trav\'es de decantaci\'on.

\noindent
\justify

El \textbf{extracto}, producto de estudio del presente trabajo, se compone de metabolitos y \textit{flavonoides} $\rightarrow$ sustancias de alto peso molecular (no vol\'atiles). Los flavonoides procedentes de algunas especies end\'emicas de la regi\'on (entre ellas, el g\'enero \textit{Lippia}, de la familia Verbenaceae) presentan diferentes propiedades de inter\'es en la medicina tradicional colombiana\footnote{Stashenko, E.E. \textit{et al}. Chromatographic and mass spectrometric characterization of essential oils and extracts from \textit{Lippia} (Verbenaceae) aromatic plants.}, por lo que representan una interesante oportunidad de innovaci\'on en la elaboraci\'on de productos farmac\'euticos de alto impacto en el mundo$^{\cite{Hennebelle2008}}$, por citar una de sus m\'ultiples aplicaciones. Debido a ello, se han desarrollado estudios referentes a la actividad biol\'ogica de extractos de diferentes especies vegetales (existen m\'as de un mill\'on de art\'iculos cient\'ificos en la base de datos de \textit{science direct} con las palabras clave \textit{natural extracts}); naciendo de all\'i la necesidad de desarrollar plantas de extracci\'on para suplir la demanda en auge de ingredientes naturales. 

\subsection{Planta desarrollada}

\noindent
\justify

Arg\"uello, J.D. \textit{et al} patentaron una planta de extracci\'on para la producci\'on de extractos vegetales$^{\cite{Patente2018}}$. La invenci\'on consiste de un molino de bolas, redise\~nado como recipiente a presi\'on, una unidad evaporadora, un condensador, una torre de enfriamiento, una bomba centr\'ifuga, mecanismo de calentamiento por resistencia el\'ectrica, una bomba de vac\'io, un compresor de aire y un filtro microm\'etrico.

\noindent
\justify

El proceso productivo de la invenci\'on (ver Figura \ref{planta}) consiste en lo siguiente: al material vegetal seco y post-destilado se le realiza un proceso de \textit{molienda} con la ayuda de un agente dispersante (material abrasivo, normalmente arena de r\'io), disminuyendo el tama\~no de part\'icula del material vegetal. A esta mezcla s\'olida entre el material vegetal pulverizado y la arena de r\'io se eluye un solvente, produci\'endose el fen\'omeno de extracci\'on mostrado en la Figura \ref{plant_extraction} y dando paso a la etapa de \textbf{eluci\'on y filtrado}. El molino de bolas se adapta, con la ayuda de un mecanismo rotatorio, para dar paso a la separaci\'on de la mezcla heterog\'enea s\'olido - l\'iquida. La mezcla se presuriza con la ayuda del compresor de aire y esta se separa a trav\'es del filtro microm\'etrico en las fases s\'olida y l\'iquida de la mezcla. La fase l\'iquida consiste en la mezcla homog\'enea entre el solvente y el extracto. Finalmente, se emplea un sistema de evaporaci\'on al vac\'io para separar el solvente del extracto (producto final). 

\begin{figure}[h!]
	\centering
	\includegraphics[width=\textwidth]{Images/Planta.png}
	\caption{Proceso productivo de la planta de extracci\'on.}
	\label{planta}
\end{figure}

\noindent
\justify

De la Figura \ref{planta}: $m_v$ es la materia prima vegetal, el agente dispersante se trata de un material auxiliar altamente abrasivo (en la mayor\'ia de aplicaciones, se trata de arena de r\'io), $m_p$ es la mezcla entre material vegetal pulverizado y el agente dispersante, $FS$ se trata del residuo s\'olido y $FL$ es la fase l\'iquida $\rightarrow$ mezcla homog\'enea entre el solvente y el extracto.

\noindent
\justify

En pruebas experimentales desarrolladas a la planta construida, se identific\'o que la etapa de eluci\'on y filtrado es el \textit{cuello de botella} del proceso productivo; debido a que el tiempo de esta etapa es cerca del doble del tiempo requerido por las otras etapas. Por esta raz\'on, en este trabajo se eval\'ua la viabilidad de un sistema de sedimentaci\'on de placas paralelas como sistema de eluci\'on y filtrado de la planta de extracci\'on; desarrollando, adem\'as, una metodolog\'ia de dise\~no autom\'atico a trav\'es de herramientas de c\'odigo abierto. Esta metodolog\'ia automatiza el proceso de dise\~no empleando m\'etodos num\'ericos (cap\'itulo \ref{CFD-DEM}) que permiten predecir el comportamiento fluido-part\'icula durante la separaci\'on de las fases s\'olida y l\'iquida. Se validad\'o la metodolog\'ia mediante un estudio experimental reportado en la literatura (cap\'itulo \ref{validacion}). 

\noindent
\justify

El presente trabajo se \textbf{organiz\'o} de la siguiente manera:

\begin{table}[h!]
	\centering
	\begin{tabular}{|c|c|p{6.5cm}|}
		\hline
		\textbf{N\'umero del cap\'itulo} & \textbf{Nombre del cap\'itulo} & \textbf{Objetivo} \\ \hline
		1 & Planta de extracci\'on & Cap\'itulo introductorio en donde se explica brevemente el caso de estudio y la raz\'on por la que se desarroll\'o el presente trabajo. \\ \hline
		2 & M\'etodo MSPD & M\'etodo ampliamente usado a escala de laboratorio para la extracci\'on de analitos de inter\'es de extractos procedentes de muestras biol\'ogicas. La planta de extracci\'on desarrollada fue dise\~nada con base en este m\'etodo experimental. \\ \hline
		3 & CFD & \multirow{3}{*}{\begin{tabular}[c]{@{}l@{}}Cap\'itulos en donde se establecen\\ las bases te\'oricas para el desarrollo\\ de las simulaciones num\'ericas.\end{tabular}} \\ \cline{1-2}
		4 & DEM & \\ \cline{1-2}
		5 & CFD-DEM & \\ \hline
		6 & Sedimentaci\'on & Cap\'itulo en donde se exponen las bases te\'oricas de la sedimentaci\'on. \\ \hline
		7 & Dise\~no te\'orico & \textbf{Propone una geometr\'ia inicial} de estudio para el desarrollo de la metodolog\'ia de dise\~no con base en simulaciones num\'ericas. \\ \hline
		8 & Modelo CFD-DEM & Desarrollo de la metodolog\'ia de dise\~no basada en el m\'etodo CFD-DEM. Permite predecir el comportamiento fluido-part\'icula del sistema. \\ \hline
		9 & Implementaci\'on del modelo & Bases del desarrollo de la metodolog\'ia de dise\~no autom\'atico mediante herramientas de c\'odigo abierto. \\ \hline
		10 & Validaci\'on del modelo & Uso del modelo CFD-DEM desarrollado en el cap\'itulo 8 para la soluci\'on de un problema reportado en la literatura que emplea experimentaci\'on. \\ \hline
	\end{tabular}
	\caption{Organizaci\'on de la tesis.}
	\label{organi}
\end{table}
\begin{center}
	\section{Introducci\'on}
\end{center}

\noindent
\justify

El m\'etodo de \textit{dispersi\'on de la matriz en fase s\'olida}, MSPD, es un m\'etodo de extracci\'on ampliamente usado a escala de laboratorio para la obtenci\'on y an\'alisis de la actividad biol\'ogica de extractos. Consiste de tres etapas: pretratamiento; eluci\'on y filtrado; y separaci\'on de sustancias. El \'exito de este m\'etodo extractivo recae en su simplicidad, rapidez y econom\'ia$^{\cite{barker2007}}$; razones por las que se han desarrollado estudios de escalabilidad en busca de la industrializaci\'on$^{\cite{Proyecto, Stashenko2017, Patente2018}}$. En estos estudios se ha reportado un cuello de botella\footnote{En el prototipo desarrollado, se evidenci\'o que la etapa de filtrado requiere de un tiempo cercano al doble en comparaci\'on con las otras etapas del proceso de extracci\'on.} durante la etapa de filtrado. Debido a ello, la presente investigaci\'on desarrolla una metodolog\'ia de dise\~no enfocada en la etapa de filtrado de una planta de extracci\'on basada en el m\'etodo MSPD; en d\'onde se aprecia en detalle la concentraci\'on de part\'iculas a lo largo del sistema de eluci\'on y filtrado, permitiendo la optimizaci\'on del mismo a trav\'es de un modelo num\'erico validado mediante experimentaci\'on.

\noindent
\justify

En general, los m\'etodos num\'ericos son teoremas matem\'aticos que permiten describir la naturaleza de diferentes fen\'omenos de car\'acter f\'isico-qu\'imico. Son ampliamente usados en ingenier\'ia como metodolog\'ias predictivas durante el proceso de dise\~no funcional y mec\'anico. Para el an\'alisis de comportamientos fluidodin\'amicos de part\'iculas, es com\'un encontrar estudios que combinen los m\'etodos num\'ericos de \textit{elementos discretos} y \textit{vol\'umenes finitos}, o como es mejor conocido: \textit{modelo CFD-DEM}.

\noindent
\justify

El acoplamiento entre CFD-DEM se ha empleado cuando se busca desarrollar an\'alisis de part\'iculas y su interacci\'on en medios viscosos. Ampliamente usado para an\'alisis de lecho fluidizado$^{\cite{Alobaid2013}}$, separadores de cicl\'on (Chu \textit{et al.}, 2009) y para el estudio de retenci\'on de part\'iculas en medios filtrantes$^{\cite{Yue2016}}$, por citar algunos ejemplos. Se han desarrollado estudios experimentales que corroboran la efectividad y viabilidad de las simulaciones num\'ericas que emplean CFD-DEM$^{\cite{Alobaid2013, Varas2017}}$.

\noindent
\justify


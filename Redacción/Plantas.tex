\subsubsection{Plantas end\'emicas}

\noindent
\justify

El tipo de especie define la actividad biol\'ogica del extracto obtenido. Un resumen de las plantas end\'emicas que se procesar\'ian en la biof\'abrica, se puede apreciar en el Cuadro \ref{plantas}.

%\begin{landscape}
\begin{table}[htp!]
\centering
\begin{adjustbox}{max width=0.94\textwidth}
\begin{tabular}{ | l | l | l | l | }
\hline
	\textbf{Especie vegetal} & \textbf{Actividad biol\'ogica} & \textbf{Composici\'on Qu\'imica} & \textbf{Aplicaciones} \\ \hline
	\multirow{2}{*}{\textit{Lippia origanoides}} & \begin{tabular}[c]{@{}l@{}}Actividad \\ antimicrobiana\end{tabular} & \begin{tabular}[c]{@{}l@{}}Carvacrol, \\
$\gamma$-terpineno,\\
timol\end{tabular} & \begin{tabular}[c]{@{}l@{}}Alta actividad \\ antioxidante 
y efectividad \\ contra control de plagas.\end{tabular} \\ \cline{2-4}
	 & Actividad antif\'ungica & \begin{tabular}[c]{@{}l@{}}Timol, O-acetiltimol, \\
O-cimeno\end{tabular} & \begin{tabular}[c]{@{}l@{}}Efectivo contra \\
P. cinnamomi.\end{tabular} \\ \hline
	\textit{Lippia canescens} & \begin{tabular}[c]{@{}l@{}}Baja actividad \\
antiproliferativa \end{tabular}& Flavonoides: Flavonas & N/A \\ \hline
	\multirow{5}{*}{\textit{Turnera diffusa}} & \begin{tabular}[c]{@{}l@{}}Actividad \\antibacteriana\end{tabular} & Extracto de hexano & \begin{tabular}[c]{@{}l@{}}Efectivo contra bacterias \\
grampositivas y \\ gramnegativas\end{tabular} \\ \cline{2-4}
	 & N/A & \begin{tabular}[c]{@{}l@{}}\'oxido de cariofileno,\\
cariofileno,\\
y-cadineno,\\
elemeno,\\
1,8-cineol\end{tabular} & \begin{tabular}[c]{@{}l@{}}Se utiliza como brebaje,\\
ingrediente para licores\end{tabular} \\ \cline{2-4}
	 & N/A & N/A & \begin{tabular}[c]{@{}l@{}}Tratamiento de\\ \'ulceras
g\'astricas. \end{tabular}\\ \hline
	\multirow{3}{*}{\textit{Cordia curassavica}} & \begin{tabular}[c]{@{}l@{}}Actividad \\antibacteriana \\y
antif\'ungica\end{tabular} & \begin{tabular}[c]{@{}l@{}}4-methyl, 4-ethenyl-3-\\
(1-methyl ethenyl)-1-\\
(1-methyl methanol)\\
cyclohexano,\\
$\beta$-Eudesmol,\\
Spathulenol,\\
Cadina\end{tabular} & \begin{tabular}[c]{@{}l@{}}Tratamiento de enferme-\\
dades infecciosas.\end{tabular} \\ \cline{2-4}
	 & Potencial larvicida & \begin{tabular}[c]{@{}l@{}}$\alpha$-pineno,\\
$\beta$-pineno,\\
E-cariofileno,\\
bicyclogermacrene\end{tabular} & \begin{tabular}[c]{@{}l@{}}Efectivo contra larvas del\\
mosquito Ae. Aegypti que\\
transmite el dengue y la \\
fiebre amarilla\end{tabular} \\ \hline
	\multirow{2}{*}{\textit{Psidium sartorianum}} & Actividad antif\'ungica & N/A & N/A \\ \cline{2-4}
	 & \begin{tabular}[c]{@{}l@{}}Actividad \\antiparasitaria\end{tabular} & Pinosotrobin chalcone & N/A \\ \hline
	\textit{Tagetes caracasana} & Actividad antiproliferante & N/A & \begin{tabular}[c]{@{}l@{}}Present\'o efecto nocivo\\ 
contra las c\'elulas \\
cancer\'igenas. \end{tabular} \\ \hline
	\multirow{2}{*}{\textit{Wedelia calycina}} & Actividad antibacteriana & \begin{tabular}[c]{@{}l@{}}Germacren-D,\\
3,3,7,7-tetrametil-5-\\
(2-metil1-propenil)-\\
triciclo [4.1.0.0(2,4)]\\
heptano,\\
$\beta$-sesquifelandreno\end{tabular} & Tratamiento contra la tos. \\ \cline{2-4}
	 & Actividad larvicida & N/A & \begin{tabular}[c]{@{}l@{}}Efectivo contra Aedes \\
aegypti\end{tabular} \\ \hline
	\multirow{3}{*}{\textit{Piper cumanense}} & Actividad antiparasitaria & N/A & \begin{tabular}[c]{@{}l@{}}Tratamiento contra la\\
malaria y la fiebre.\end{tabular} \\ \cline{2-4}
	 & Actividad antif\'ungica & \multirow{2}{*}{\begin{tabular}[c]{@{}l@{}}$\alpha$-pineno,\\
$\beta$-pineno,\\
linanool,\\
germacreno,\\
$\beta$-cariofileno\end{tabular}} & \begin{tabular}[c]{@{}l@{}}Efectivo contra Fusarium \\
oxysporum.\end{tabular} \\ \cline{2-2} \cline{4-4}
	 & Actividad insecticida &  & \begin{tabular}[c]{@{}l@{}}Efectivo contra Sitophilus\\
zeamis y Spodoptera \\
frugiperda. \end{tabular}\\ \hline
\end{tabular}
\end{adjustbox}
\caption{Resumen de algunas plantas end\'emicas de Colombia.}
\label{plantas}
\end{table}
%\end{landscape}
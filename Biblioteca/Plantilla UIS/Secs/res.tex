% ------------------------------------------------------------------------
% ------------------------------------------------------------------------
% ------------------------------------------------------------------------
%                                Resumen
% ------------------------------------------------------------------------
% ------------------------------------------------------------------------
% ------------------------------------------------------------------------
\chapter*{RESUMEN}

\footnotesize{
\begin{description}
  \item[T�TULO:] VERIFICACI�N DE CONJUNTOS ESTABILIZANTES PARA EL M�TODO DE DISE�O DE CONTROLADORES PI DE ZIEGLER \& NICHOLS \astfootnote{Trabajo de grado}
  \item[AUTOR:] EMERSON REY ARDILA \asttfootnote{Facultad de Ingenier�as F�sico-Mec�nicas. Escuela de Ingenier�as El�ctrica, Electr�nica y telecomunicaciones. Director: Ricardo Alzate Casta�o, Doctorado en Ingenier�a Inform�tica y Autom�tica.}
  \item[PALABRAS CLAVE:] CONJUNTO ESTABILIZANTE, CONTROLADORES PI, DISE�O GR�FICO DE COMPENSADORES, M�TODO DE ZIEGLER \& NICHOLS.
  \item[DESCRIPCI�N:]\hfill \\ El presente proyecto de grado presenta el c�lculo de conjuntos estabilizantes para
sistemas SISO LTI controlados por compensadores de estructura simple. En particular, se estudia la fragilidad de controladores PI calculados empleando el m�todo cl�sico de \emph{Ziegler \& Nichols} empleado como t�cnica de referencia en m�ltiples aplicaciones de ambito industrial. A partir de la definici�n para una m�trica basada en la interpretaci�n geom�trica para los m�rgenes de estabilidad del sistema controlado, se verifica que el controlador dise�ado con el m�todo en cuesti�n no necesariamente tolera variaciones significativas en sus valores de par�metro. Por el contrario, asume comportamientos cercanos a los l�mites de estabilidad proporcionados mediante el c�lculo de su conjunto estabilizante. Lo anterior se convierte en informaci�n importante tomando en cuenta que generalmente los m�todos de dise�o se someten a un ajuste fino. Como m�trica, se define el espacio planar correspondiente con la intersecci�n entre una elipse y una l�nea recta que representan lugares geom�tricos de m�rgen de fase y/o ganancia constantes. Adicional a lo anterior, se desarroll� una interfaz en MATLAB que permite calcular gr�ficamente los par�metros del controlador a partir de un conjunto admisible de especificaciones con base en su conjunto estabilizante. Trabajo complementario incluye la utilizaci�n de t�cnicas computacionales para el c�lculo de
conjuntos estabilizantes sobre plantas arbitrarias.
\end{description}}\normalsize
% ------------------------------------------------------------------------ 
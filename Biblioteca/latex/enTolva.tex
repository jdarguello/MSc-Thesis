\subparagraph{Tuber\'ia de entrada y tolva de sedimentos}

\noindent
\justify

El di\'ametro de entrada de la mezcla l\'iquido-s\'olido se calcula de la siguiente forma:

\begin{equation}
	D_e = \frac{4 Q}{Re \, \pi \, \mu}
	\label{De}
\end{equation}

\noindent
\justify

De la Ecauci\'on \ref{De}: $D_e$ es el di\'ametro de la tuber\'ia de entrada del sistema de sedimentaci\'on, $Q$ es el caudal de total de entrada, $Re$ es el n\'umero de Reynolds del flujo de entrada $( Re \geq 250)$ y $\mu$ es la viscocidad cinem\'atica del fluido.

La tolva de sedimentos cumple el objetivo de almacenar los lodos obtenidos durante el proceso de sedimentaci\'on. El di\'ametro m\'inimo de desag\"ue es de $30 [cm]^{\cite{RobertoRojas}}$ y la pendiente longitudinal var\'ia entre el $2$ y $3 \%$. El tanque debe tener la capacidad para ser desocupado hasta en un m\'aximo de $60 [min]$. La tuber\'ia de desag\"ue se puede calcular con base en la Ecuaci\'on \ref{Des}${\cite{RobertoRojas}}$.

\begin{equation}
	S = \frac{A}{4850 \, t} \sqrt{d}
	\label{Des}
\end{equation}

\noindent
\justify

De la Ecuaci\'on \ref{Des}: $S$ es la secci\'on del desag\"ue, en $m^2$, $A$ es el \'area superficial del sedimentador, $m^2$, $t$ es el tiempo de vaciado, en $h$ y $d$ es la altura del agua sobre la boca del desag\"ue, $m$.

\noindent
\justify

El caudal de descarga se se calcula de la siguiente forma:

\begin{equation}
	Q = 0.61 \, S \, \sqrt{2 \, g \, d}
	\label{Qdes}
\end{equation}

\noindent
\justify

Para evitar el asentamiento de lodos en la tuber\'ia de desag\"ue, la velocidad de flujo debe ser mayor de $1.4 [m/s]^{\cite{RobertoRojas}}$.
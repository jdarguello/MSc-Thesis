\subsection{Selecci\'on del m\'etodo de extracci\'on}

\noindent
\justify

El Cuadro \ref{compatec} presenta un resumen comparativo de las t\'ecnicas de extracci\'on explicadas en la secci\'on 4.2.

\begin{table}[h!]
\centering
\begin{adjustbox}{max width=0.99\textwidth}
\begin{tabular}{|c|l|l|p{3cm}|l|}
\hline
\textbf{Par\'ametro} & \textbf{MSPD} & \textbf{UAE} & \textbf{S\'olido - L\'iquido} & \textbf{SFE} \\ \hline
Ventajas & \begin{tabular}[c]{@{}l@{}} - Alta eficiencia de\\  extracci\'on (entre el \\ 20 y 30\%). \\ - Bajo costo de\\ materia prima y \\ energ\'ia. \\ - Permite el uso\\ de una amplia\\ gama de solventes. \end{tabular} & \begin{tabular}[c]{@{}l@{}} - Debido a las\\ ondas de choque en \\ un medio con\\ solvente, se \\ produce tanto \\ la disrupci\'on como\\ la transferencia de \\ masa.   \end{tabular} & \begin{tabular}[c]{@{}l@{}} - Existen patentes \\a escala industrial \\ - Permite el uso\\ de una amplia \\ gama de solventes \end{tabular} & \begin{tabular}[c]{@{}l@{}} - La separaci\'on \\ entre el solvente \\ $(CO_2)$ y el \\ extracto es\\ instant\'anea. \\ - Existen patentes \\ a escala industrial.  \end{tabular} \\ \hline
Desventajas & \begin{tabular}[c]{@{}l@{}} - Requiere de un \\ proceso posterior \\ de separaci\'on \\ de sustancias. \\ - No existen plantas \\ de extracci\'on \\ a escala industrial \\ que empleen este \\ m\'etodo\footnote{\textit{Oportunidad de innovaci\'on.}}. \end{tabular} & \begin{tabular}[c]{@{}l@{}} - Requiere de un alto \\ consumo energ\'etico. \\- Requiere de un \\ proceso posterior \\ de separaci\'on \\ de sustancias. \end{tabular} & \begin{tabular}[c]{@{}l@{}} - Presenta una \\eficiencia de \\extracci\'on baja \\en hojas, ramas \\y ra\'ices$^{\cite{Liadakis2003}}$. \end{tabular} & \begin{tabular}[c]{@{}l@{}} - Requiere de un alto \\ consumo energ\'etico. \\ - Bajo rendimiento \\ de extracci\'on. \\ - S\'olo permite la\\ obtenci\'on de un\\ tipo espec\'ifico\\ de compuestos\\ debido a la\\ polaridad del\\ solvente. \end{tabular} \\ \hline
\end{tabular}
\end{adjustbox}
\caption{Cuadro comparativo de t\'ecnicas de extracci\'on.}
\label{compatec}
\end{table} 

\noindent
\justify

Debido al an\'alis desarrollado, y mostrado en el Cuadro \ref{compatec}, el m\'etodo base seleccionado para la planta de extracci\'on es el \textbf{m\'etodo MSPD}. Como se observa en las Figuras \ref{cadena} y \ref{mspd}, la planta requiere de tres etapas para la producci\'on de extracto:

\begin{enumerate}
	\item \textit{Preprocesamiento:} disminuye la granulometr\'ia del material vegetal. Tiene la funci\'on de incrementar el \'area de transferencia de masa para mejorar la eficiencia del proceso extractivo.
	\item \textit{Eluci\'on y filtrado:} el material particulado entra en contacto con un solvente. La polaridad de este tiene la capacidad de atraer los metabolitos y flavonoides de inter\'es. A trav\'es de un proceso de filtrado, se obtiene una mezcla s\'olido - l\'iquido entre el extracto y el solvente.
	\item \textit{Separaci\'on de sustancias:} tiene el objetivo de recuperar el solvente empleado, para su reutilizaci\'on, sin degradar el extracto.
\end{enumerate}
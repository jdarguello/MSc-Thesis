\subsection{M\'etodo de Dispersi\'on de la Matriz en Fase S\'olida}	\label{MSPD_chap}

\noindent
\justify

El m\'etodo de dispersi\'on de la matriz en fase s\'olida (MSPD, por sus siglas en ingl\'es) ha sido ampliamente utilizado para el estudio de muestras biol\'ogicas$^{\cite{barker2007, Capriotti2015, Perez2019}}$. Existen m\'as de 250 publicaciones en las que se emplea este m\'etodo extractivo para el an\'alisis de extractos de distintas naturalezas $^{\cite{barker2007}}$. Esto se debe a la alta eficiencia y bajo costo de este m\'etodo de extracci\'on. 

\noindent
\justify

Consiste, b\'asicamente, de tres etapas (como se puede observar en la Figura \ref{mspd}):

\begin{enumerate}
	\item Maceraci\'on de la muestra con un \textit{agente dispersante} (material particulado, normalmente compuesto de s\'ilice).
	\item Homogenizaci\'on de la muestra macerada en la columna.
	\item Eluci\'on con solvente y filtrado de la mezcla \textit{solvente - extracto}.
\end{enumerate}

\begin{figure}[h!]
\centering
\includegraphics[width=0.8\textwidth]{Images/mspd.PNG}
\caption{M\'etodo MSPD$^{\cite{Capriotti2015}}$.}
\label{mspd}
\end{figure}

\subsubsection{Factores a considerar en la extracci\'on MSPD}

\noindent
\justify

Hay varios factores a considerar en la extracci\'on MSPD, que incluye:

\begin{enumerate}
	\item \textit{Efecto del tama\~no de part\'icula media:} tama\~nos de part\'icula peque\~nos (entre $3$ - $10 \, \mu m ^{\cite{barker2007}}$) requiere de grandes tiempos de eluci\'on y altos gradientes de presi\'on para obtener un flujo adecuado. 
	\item \textit{Agente dispersante:} el uso de silicatos infravalorados, como la arena de r\'io, para la maceraci\'on de muestras presenta resultados diferentes a los reportados con agentes dispersantes como el $C_{18}$ o el $C_8$. A pesar de que el mismo principio de disrupci\'on de la matriz se conserva, debido a la abrasi\'on, es probable que se de una interacci\'on qu\'imica no deseada entre silicatos infravalorados y algunos de los flavonoides del extracto.
	\item \textit{Relaci\'on m\'asica:} la mejor relaci\'on m\'asica reportada en la literatura frecuenta ser una relaci\'on 1 a 4 $^{\cite{barker2007}}$, aunque puede variar de una aplicaci\'on a otra. 
	\item \textit{Solvente:} el vertimiento del solvente en la columna MSPD tiene el fin de aislar analitos espec\'ificos o familias de compuestos. El tipo de solvente, y la polaridad de este, define la composici\'on final del extracto. Existen estudios en donde se ha demostrado un incremento en el rendimiento extractivo al emplear solventes a temperaturas superiores a la temperatura ambiente e inferiores a los $60 \left[ \degree C \right]^{\cite{Vieira2019}}$. 
\end{enumerate}

\subsubsection{Extracci\'on en fase s\'olida}

\noindent
\justify

El m\'etodo MSPD presenta diferencias claras respecto a la extracci\'on fase s\'olida cl\'asica (SPE, por sus siglas en ingl\'es); entre ellas$^{\cite{barker2007}}$:
\begin{enumerate}
	\item Al emplear el m\'etodo MSPD, se consigue una disrupci\'on completa de la muestra en part\'iculas de reducido tama\~no, incrementando el \'area de extracci\'on. En SPE, la disrupci\'on de la muestra se considera un paso \textit{adicional}, donde muchos de los compuestos se descartan al procesar la muestra para la columna SPE. 
	\item En SPE, la muestra es usualmente absorbida en la parte superior de la columna y no a trav\'es de ella, como en el m\'etodo MSPD.
	\item La interacci\'on f\'isica y qu\'imica de los compuestos del sistema son mayores en el m\'etodo MSPD y diferentes, en diversos sentidos, de aquellos apreciados en el SPE cl\'asico, incluyendo otras formas de cromatograf\'ia l\'iquida.
\end{enumerate}
\begin{center}
	\section{Formulaci\'on de la Pregunta e Hip\'otesis}
\end{center}

\subsection{Pregunta de investigaci\'on}

\noindent
\justify

?`Qu\'e nivel de precisi\'on se podr\'ia obtener al implementar una metodolog\'ia de dise\~no que emplee los m\'etodos de \textit{elementos discretos} (DEM) y \textit{vol\'umenes finitos} (FVM), para el an\'alisis del comportamiento fluidodin\'amico durante la etapa de filtrado de una planta de extracci\'on basada en el m\'etodo de dispersi\'on de la matriz en fase s\'olida - MSPD?

\subsection{Hip\'otesis}

\noindent
\justify

Se espera alcanzar un nivel de precisi\'on igual o superior al $75\%$ al comparar los resultados de la simulaci\'on num\'erica con respecto a los obtenidos por un estudio experimental sobre un sistema f\'isico real, debido a que la correlaci\'on entre los m\'etodos num\'ericos DEM y FVM permite describir el comportamiento fluidodin\'amico de part\'iculas en medios viscosos.
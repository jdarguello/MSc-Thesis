\begin{center}
	\section{Conclusiones}
\end{center}

\noindent
\justify

Con base en los resultados obtenidos, se puede apreciar la importancia de las simulaciones num\'ericas basadas en CFD-DEM para el dise\~no de maquinaria y equipos que busquen separar mezclas de sustancias s\'olidas y l\'iquidas. De la Figura \ref{CFDEM:fpart} se concluye que el panel de lamelas presenta zonas de salidas \textit{limpias} de material particulado y otras con una concentraci\'on menor a la inicialmente procesada. Futuras investigaciones podr\'ian emplear el modelo CFD-DEM desarrollado en el presente trabajo como base para el dise\~no de un nuevo sistema de sedimentaci\'on que emplee un \textit{reflujo} de las zonas con posibles impurezas para garantizar una completa separaci\'on de sustancias.

\noindent
\justify

Se logr\'o implementar la metodolog\'ia del modelo CFD-DEM con la ayuda de herramientas de c\'odigo abierto; demostrando la viabilidad de estas en el desarrollo de nuevos productos y servicios de bajos costos de inversi\'on inicial y alto grado de innovaci\'on.

\noindent
\justify

A las condiciones de operaci\'on dadas ($0.3 \left[ m^3/h \right]$ de caudal, $27 \left[ \degree C \right]$ de temperatura de operaci\'on y $250 [ \mu m]$ de tama\~no de part\'icula medio) las simulaciones num\'ericas del sistema de eluci\'on y filtrado demostraron un nivel de eficiencia general superior al $90 \%$ en la remoci\'on de material particulado. Estos resultados presentan un margen de error inferior al $3.18 \%$ gracias a la caracterizaci\'on de la malla empleada durante el an\'alisis del error computacional (secci\'on \ref{CompiError}); en d\'onde se concluy\'o que la malla m\'as adecuada para el desarrollo de la simulaci\'on CFD-DEM del sistema de sedimentaci\'on se trataba de una malla con elementos rectangulares de $17636$ nodos, tama\~no m\'aximo de elementos de $10 [mm]$ y con un refinamiento en la zona de lamelas con elementos de $5 [mm]$ de tama\~no.

\noindent
\justify

Con base en los resultados obtenidos, se concluye que el sistema de sedimentaci\'on es una soluci\'on viable para emplearlo como sistema de separaci\'on de sustancias de la planta de extracci\'on a las condiciones de dise\~no especificadas en el Cuadro \ref{condiciones}.

\noindent
\justify

De acuerdo al proceso de validaci\'on desarrollado en el cap\'itulo \ref{validacion}, se concluye que el modelo CFD-DEM desarrollado es preciso y confiable (error inferior al $2\%$ con respecto a los resultados experimentales) y puede ser empleado en diversas aplicaciones de flujo en donde part\'iculas s\'olidas interact\'uan con fluidos newtonianos en fase l\'iquida o gaseosa; gracias a que permite apreciar la interacci\'on fluido-part\'icula durante la separaci\'on de sustancias.

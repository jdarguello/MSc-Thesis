\begin{center}
	\section{Recomendaciones}
\end{center}

\noindent
\justify

Al dise\~no final del sistema se recomienda emplear un sistema de reflujo en la \'ultima lamela para garantizar una completa separaci\'on de la fase s\'olida y l\'iquida.

\noindent
\justify

Es recomendable construir un prototipo del sistema de sedimentaci\'on dise\~nado para ejecutar un dise\~no experimental con las variables descritas en el Cuadro \ref{exps}.

\begin{table}[h!]
	\centering
	\begin{tabular}{|c|c|}
		\hline
		\textbf{Variable} & \textbf{Rango} \\ \hline
		Temperatura & $[20 \, \degree C, 60 \, \degree C]$ \\ \hline
		Tiempo de separaci\'on $[h]$ & $[1, 4]$ \\ \hline
		Mezcla agua - etanol $[\% ]$ & $(0,25,50,75,100)$ \\ \hline
	\end{tabular}
	\caption{Dise\~no experimental sugerido.}
	\label{exps}
\end{table}

\noindent
\justify

Una vez seleccionada una especie vegetal de estudio, las variables de respuesta a analizar deber\'ian ser: rendimiento de extracci\'on (valor adimensional que contrasta la cantidad de material procesado con la cantidad de producto obtenido) y la composici\'on qu\'imica del extracto, obtenida mediante cromatograf\'ia l\'iquida (permite conocer la calidad del extracto).

\noindent
\justify

El software desarrollado, y documentado en el Anexo A, se puede complementar con m\'odulos de dibujo CAD a la geometr\'ia final (tanto 2D como 3D) e informes de ingenier\'ia autom\'atico en \LaTeX. Debido al hecho de haber sido desarrollado \'unicamente con lenguajes y herramientas de c\'odigo abierto, es posible crear un servicio de dise\~no web interactivo que resuelva simulaciones num\'ericas en la nube.
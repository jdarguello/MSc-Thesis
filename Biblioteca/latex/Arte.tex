\begin{center}
	\section{Estado del Arte}
\end{center}

\noindent
\justify

En la presente secci\'on, se muestra un resumen de diferentes avances cient\'ificos, ordenados cronol\'ogicamente y mediante normativa IEEE, de los siguientes temas: MSPD, simulaci\'on num\'erica en sistemas de extracci\'on, sistemas de molienda y peletizado, y patentes de sistemas de extracci\'on.

\begin{itemize}
	\item{\textbf{Yao, L.M. \textit{et al. An optimized CFD-DEM method for fluid-particle coupling dynamics analysis}. International Journal of Mechanical Sciences. 2020. \\https://doi.org/10.1016/j.ijmecsci.2020.1055034}}
\end{itemize}

\noindent
\justify

Al emplear el m\'etodo CFD-DEM en la soluci\'on de flujos de dos fases (compuesto por fluido fracturado y arena de cuarzo) a trav\'es de tuber\'ias, a pesar de que las colisiones y acumulaci\'on de part\'iculas se pueden describir a escala mesosc\'opica, presenta serios problemas de eficiencia y precisi\'on. En el presente estudio, se plantea un m\'etodo CFD-DEM mejorado basado en incrementos de tiempo de rotonda. Al implementar la mejora en la estrategia de soluci\'on del problema CFD-DEM en t\'erminos del tiempo creciente, el criterio de convergencia iterativo y el algoritmo de avance de tiempo de la pareja fluido-part\'icula se establecieron; ajustando de manera autom\'atica la los pasos de tiempo del modelo de acuerdo al criterio de convergencia.

\begin{itemize}
	\item{\textbf{M. N. Alnajrani and O. A. Alsager. \textit{Removal of antibiotics from water by polymer of intrinsic microporosity: isotherms, kinetics, thermodynamics and adsorption mechanism}. Nature Research. 2020. \\https://doi.org/10.1038/s41598-020-57616-4}}
\end{itemize}

\noindent
\justify

Investigaci\'on acerca de la capacidad filtrante de un pol\'imero de microporosidad intr\'inseca (PIM) para la remoci\'on de antibi\'oticos presentes en aguas residuales como la doxiciclina, ciproflaxina, penicilina G y amoxicilina; llegando a remover cerca del $80\%$ de la concentraci\'on inicial. Se emplearon los modelos de Langmuir y Freundlich para la correlaci\'on de los datos de equilibrio. Las constantes de adsorci\'on se evaluaron a partir de pseudo modelos de segundo orden, y los par\'amtros termodin\'amicos se obtuvieron mediante los estudios de adsorci\'on a diferentes temperaturas de reacci\'on.

\begin{itemize}
	\item{\textbf{Avi Uzi, G. B. Halevy y Avi Levy. \textit{CFD-DEM Modeling of soluble NaCl particles conveyed in brine}. Powder Technology. 2020. \\https://doi.org/10.1016/j.powtec.2019.09.039}}
\end{itemize}

\noindent
\justify

Este estudio se enfoca en la disoluci\'on de cloruro de sodio en salmuera empleando CFD-DEM en una escala unidimensional. Diversas funciones de usuario han sido integradas en el c\'odigo para describir las propiedades termof\'isicas variacionales con el cambio de concentraci\'on. El modelo completo es comparado con diferentes experimentos estudiando dos p\'arametros clave: tama\~no de part\'icula y concentraci\'on de la salmuera. La precisi\'on se estableci\'o a partir del n\'umero de Sherwood para la correlaci\'on de la transferencia de masa convectiva compar\'andola con la experimentaci\'on f\'isica.

\begin{itemize}
	\item{\textbf{E. O. dos Santos, \textit{et al.} \textit{Sand as a solid support in ultrasound-assisted MSPD: A simple, green and low-cost method for multiresidue pesticide determination in fruits and vegetables}. Food Chemistry. 2019. \\doi: https://doi.org/10.1016/j.foodchem.2019.05.200}}
\end{itemize}

\noindent
\justify

Art\'iculo donde se estudi\'o la viabilidad de emplear arena como soporte s\'olido en ultrasonido asistido con dispersi\'on de la matriz en fase s\'olida (UA-MSPD) para la extracci\'on de diferentes clases de pesticidas, incluyendo: organofosforados, carbamatos, triazonas y piratoides de frutas y vegetales, determinados por las t\'ecnicas GC-MS y LC-MS/MS. El desempe\~no de la arena se compar\'o con el de diferentes tipos de soportes s\'olidos, entre cl\'asicos y alternativos naturales. Los mejores resultados se obtuvieron con $0.5 [g]$ de muestra, $1 [g]$ de arena, $20 [mg]$ de carb\'on activado y $5 [mL]$ de acetato de etilo como solvente de eluci\'on.  El m\'etodo desarrollado fue eficiente, simple, barato, robusto y ambientalmente amigable.

\begin{itemize}
	\item{\textbf{C. Sielfeld, J. M. del Valle and F. Sastre, \textit{Effect of pelletization on supercritical $CO_2$ extraction of rosemary antioxidants}. The Journal of Supercritical fluids. pp. 162-171, 2019. \\doi: https://doi.org/10.1016/j.supflu.2016.04.010}}
\end{itemize}

\noindent
\justify

Investigaci\'on donde se estudia el efecto de peletizaci\'on de hojas de romero para el proceso de extracci\'on mediante $CO_2$ supercr\'itico $\left(SC-CO_2 \right)$. Se desarroll\'o un dise\~no de experimento factorial $\left(2^3 \right)$, donde se estudiaron los efectos referentes a la temperatura (40 y 70 $\degree$C) y presi\'on (30 y 50 MPa) de extracci\'on; y el efecto del pretratamiento del sustrato. Las condiciones en el punto central del dise\~no experimental son de 55 $\degree$C y 40 MPa, con control del sustrato para los puntos centrales de extracci\'on, debido a que el pretratamiento es una variable categ\'orica. 

\noindent
\justify

En el presente estudio, se observ\'o que el rendimiento de extracci\'on disminuy\'o del $4.28 \%$ al $1.33 \%$; el rendimiento volum\'etrico (masa de soluto recuperado por unidad de tiempo y por unidad de volumen extra\'ido) se increment\'o hasta el $34 \%$; se redujo la actividad antioxidante del extracto debido a las temperaturas presentadas durante el proceso de peletizado; y se corrobor\'o una relaci\'on lineal $\left( R^2 = 0.83 \right)$ entre la actividad antioxidante y el \'acido carn\'osico fen\'olico del romero.

\begin{itemize}
	\item{\textbf{M. E. Mostafa \textit{et al, The significance of pelletization operating conditions: an analysis of physical and mechanical characteristics as well as energy consumption of biomass pellets}. Renewable and sustainable energy reviews. pp. 332-348, 2019. \\
	doi: https://doi.org/10.1016/j.rser.2019.01.053}}
\end{itemize}

\noindent
\justify

Art\'iculo de investigaci\'on que trata sobre la importancia de las condiciones de operaci\'on para la elaboraci\'on de biomasa peletizada como combustible alternativo, para la producci\'on de energ\'ia dentro de una termoel\'ectrica. El proceso de peletizado puede realizarse tanto con una peletizadora como con un sistema cilindro - pist\'on, donde, en ambos casos, la biomasa es comprimida de tal forma que se producen peque\~nas unidades, o pellets, permitiendo tener grandes masas de material vegetal en un volumen reducido. 

\noindent
\justify

En este trabajo, se desarolla una metodolog\'ia de c\'alculo que permite definir: longitud y di\'amtro de los pellets, humedad final del material vegetal, durabilidad mec\'anica (resistencia abrasiva), dureza y consumo energ\'etico, entre muchos otros.

\begin{itemize}
	\item \textbf{T. Naukkarinen, M. Nikku y T. Turunen-Saaresti \textit{CFD-DEM simulations of hydrodynamics of combined ion exchange-membrane filtration}. Chemical Engineering Science. 2019. \\ doi: https://doi.org/10.1016/j.ces.2019.08.009}
\end{itemize}

\noindent
\justify

El sistema de filtrado combinado ion de intercambio-membrana, donde una capa de filtrado activa es formada usando una cama de resinas y agua, tiene el potencial de reducir el uso de energ\'ia e incrementar la calidad de aguas residuales purificadas. El dise\~no hidrodin\'amico de este sistema est\'a en su \textit{infancia} y se requiere del uso de m\'etodos num\'ericos para su an\'alisis. En este trabajo, se emplea el m\'etodo CFD-DEM para analizar la hidrodin\'amica de este sistema, la formaci\'on del lecho de filtrado, entro otros par\'ametros de inter\'es al variar las condiciones de flujo. Como resultado, los detalles de la formaci\'on del lecho, ideas sobre otros par\'ametros que afectan este fen\'omeno y sugerencias para mejorar el dise\~no hidrodin\'amico de este sistema han sido presentadas.

\begin{itemize}
	\item \textbf{J.D. Arg\"uello \textit{et al. Equipo, proceso y producto obtenido a partir de material vegetal con propiedades biol\'ogicas}. CO2018013023A1. Nov, 18, 2018.}
\end{itemize}

\noindent
\justify

Patente que expone una metodolog\'ia para la producci\'on de extractos vegetales a escala prototipo. Presenta los cimientos de la escalabilidad de plantas de extracci\'on basadas en el m\'etodo de \textit{dispersi\'on de la matriz en fase s\'olida} (MSPD, por sus siglas en ingl\'es).

\begin{itemize}
	\item \textbf{M. Cerqueira \textit{et al. Sampe as solid support in MSPD: A new possibility for determination of pharmaceuticals, personal care and degradation products in sewage sludge}. CO2018013023A1. Nov, 18, 2018.}
\end{itemize}

\noindent
\justify

M\'etodo de extracci\'on basado en la dispersi\'on de la matriz en fase s\'olida (MSPD) y enfocada en los principios la qu\'imica anal\'itica ambiental. Propone soportes s\'olidos y solventes de menor toxicidad para la determinaci\'on simult\'anea de 19 f\'armacos, cuatro productos de cuidado personal y cuatro m\'as de productos degradados en muestras de aguas residuales. Altas recuperaciones fueron obtenidas cuando $2 [g]$ de muestra fue macerada por $5 [min]$ en un mortero, transferido luego a un tubo centr\'ifugo, y $1 [min]$ de agitaci\'on con $5 [mL]$ de metanol. 

\begin{itemize}
	\item{\textbf{E. E. Stahenko \textit{et al. Numerical simulation through the discrete element method (DEM) and the finite volume method (FVM) of the crushing and filtering processes for the production of vegetable extracts by the matrix solid-phase dispersion (MSPD).} XI Congreso Colombiano de M\'etodos Num\'ericos. ISBN: 978-958-48-5477-3. pp. 57-67, 2018.}}
\end{itemize}

\noindent
\justify

Art\'iculo de investigaci\'on que trata sobre el an\'alisis num\'erico del dise\~no de una planta de producci\'on de extractos vegetales. La planta dise\~nada se basa en el m\'etodo MSPD y consta de un sistema de trituraci\'on, adaptado como recipiente a presi\'on, y de un sistema de recuperaci\'on del solvente. 

\noindent
\justify

En el art\'iculo, se desarrolla un estudio num\'erico mediante los m\'etodos del elemento discreto (DEM) y de vol\'umenes finitos (FVM) de los procesos de trituraci\'on y filtrado, respectivamente, donde se propone un redise\~no referente a un tama\~no \'optimo de las bolas del molino y de la presi\'on de operaci\'on durante el proceso de filtrado.

\begin{itemize}
	\item{\textbf{J. D. Arg\"uello y O. G\'omez. \textit{Dise\~no de un prototipo de una planta destinada a la producci\'on de extractos vegetales mediante el m\'etodo de extracci\'on de dispersi\'on de la matriz en fase s\'olida - MSPD, con sistema de recuperaci\'on de solvente.} Bucaramanga, Colombia: Universidad Industrial de Santander, 2017.}}
\end{itemize}

\noindent
\justify

Trabajo de grado que trata sobre el dise\~no y construcci\'on de un prototipo de una planta destinada a la producci\'on de extractos vegetales; d\'onde, adem\'as, se propone el dise\~no de una planta con capacidad de procesamiento de $100[kg/lote]$, $4$ lotes al d\'ia.

\noindent
\justify

En este trabajo, se eval\'ua la viabilidad de la escalabilidad del m\'etodo MSPD a partir de estudios experimentales sobre el prototipo construido.

\vspace{1cm}

\begin{itemize}
	\item{\textbf{C. Yue, Q. Zhang and Z. Zhai. \textit{Numerical simulation of the filtration process in fibrous filters using CFD-DEM method.} Journal of Aerosol Science, Volume 101. pp. 174-187, 2016.}}
\end{itemize}

\noindent
\justify

Art\'iculo de investigaci\'on d\'onde se desarrolla un estudio num\'erico, mediante CFD (computational fluid dynamics) y DEM (discrete element method), del proceso de filtrado de part\'iculas presentes en un flujo de aire sobre filtros fibrosos con di\'ametro de fibra entre 10 y 20 micr\'ometros. 

\begin{itemize}
	\item{\textbf{A. L. Capriotti \textit{et al. Recent advances and developements in matrix solid-phase dispersion}. Trends in analytical chemistry, Volume 71. pp. 186-193, 2015. 10.1016/j.trac.2015.03.012}}
\end{itemize}

\noindent
\justify

Rese\~na donde se presentan y analizan algunos de los \'ultimos avances referentes al m\'etodo de dispersi\'on de la matriz en fase s\'olida - MSPD. Diferentes m\'etodos de preparaci\'on de muestras, materiales inusuales e innovadores empleados como dispersantes y la combinaci\'on entre MSPD con otros m\'etodos extractivos, como la extracci\'on mediante l\'iquido presurizado, por citar un ejemplo, son algunos de los temas tratados en este art\'iculo de investigaci\'on.

\begin{itemize}
	\item{\textbf{E.E. Stashenko \textit{et al}. \textit{Chromatography and mass spectrometric characterization of essential oils and extracts from Lippia (Verbenaceae) aromatic plants.} Journal of separation science. pp. 192-202, 2013. doi: 10.1002/jssc.201200877.}}
\end{itemize}

\noindent
\justify

Art\'iculo de investigaci\'on que trata sobre el estudio de los diferentes compuestos qu\'imicos, mediante el uso de metodolog\'ias anal\'iticas GC y HPLC, presentes en hojas y ramas de plantas arom\'aticas pertenecientes al g\'enero \textit{Lippia}, familia Verbenacea, de las que destacan las especies: Lippia alba, Lippia origanoides, Lippia micromera, Lippia americana, Lippia graveolens y Lippia citridora. Se emplearon m\'etodos de extracci\'on por solvente (metanol) y fluido supercr\'itico $\left(CO_2 \right)$.

\begin{itemize}
	\item{\textbf{Liu Kuiyu \textit{et al, Equipment and method for extracting biologically active ingredients from subcritical fluid}. CN 1 019 0509 1B, dec, 08, 2010.}}
\end{itemize}

\noindent
\justify

Invenci\'on que involucra tanto la metodolog\'ia como el equipo de extracci\'on de componentes biol\'ogicos a partir de fluido supercr\'itico. El equipo consiste de un sistema de suministros, sistema de extracci\'on, sistema de separaci\'on y sistema de recuperaci\'on del solvente.

\begin{itemize}
	\item{\textbf{D. A. Fl\'orez, W. A. Ram\'irez y L. B. Varela, \textit{Dise\~no conceptual de una m\'aquina peletizadora de alimentos para aves de corral}. Medell\'in, Colombia: Universidad Nacional de Colombia, 2010.}}
\end{itemize}

\noindent
\justify

Trabajo de grado que consiste en el dise\~no conceptual de una m\'aquina peletizadora para la elaboraci\'on de alimentos para pollos, en forma de pellet. Se estudian diversas variables de dise\~no, entre ellas: tipo de peletizadora, m\'etodos de transmisi\'on de energ\'ia (t\'ermica y mec\'anica), selecci\'on de componenetes y equipos, dise\~no de ejes y paletas, y recomendaciones de mantenimiento, entre muchas otras. 

\begin{itemize}
	\item{\textbf{M. Kimura, M. Narumi and T. Kobayashi. \textit{Design method of ball mill by discrete element method.} Sumitomo Chemical Co., Ltd. pp. 1-9, 2007}}
\end{itemize}

\noindent
\justify

Se plantea una metodolog\'ia de dise\~no, mediante simulaciones num\'ericas que emplean el m\'etodo del elemento discreto, de molinos de bolas destinados a la \'optima trituraci\'on de rocas. Esta metodolog\'ia emplea como par\'ametros de entrada: velocidad de rotaci\'on, tama\~no de las bolas, propiedades f\'isicomecanicas del material a triturar, geometr\'ia de las aletas, o elevadores, y coeficientes de fricci\'on, entre otros, y entre los resultados esperados, se encuentran: granulometr\'ia final del material a triturar en funci\'on del tiempo de molienda, cinem\'atica de las bolas en funci\'on de la velocidad de rotaci\'on, relaci\'on entre energ\'ia de impacto de las bolas y granulometr\'ia final del material.

\begin{itemize}
	\item{\textbf{A. Ryser, \textit{et al. Cap extraction device}}, US 8613246B2, 2003.}
\end{itemize}

\noindent
\justify

Invenci\'on de \textit{Societe des Produits} (Nestle SA) que muestra el funcionamiento mec\'anico de un dispositivo de extracci\'on por lotes. Consta de un sistema de compresi\'on y una parte movible donde va la materia prima encapsulada para desarrollar la extracci\'on.

\begin{itemize}
	\item{\textbf{P. Mengal and B. Mompon, \textit{Method and plant for solvent-free microwave extraction of natural products}, EP 0 698 076 B1, Enero, 14, 1998.}}
\end{itemize}

\noindent
\justify

Patente que trata sobre la metodolog\'ia de extracci\'on de material biol\'ogico sin solvente mediante radiaci\'on por microondas. La exposici\'on del material a la radiaci\'on produce que se separe el producto natural deseado del residuo biol\'ogico, donde se involucra presi\'on de vac\'io durante la etapa extractiva por radiaci\'on para evitar la degradaci\'on del producto.

\begin{itemize}
	\item{\textbf{E. Favre and P. Masek. \textit{Device for the extraction of cartridges}, US4846052A, 1986.}}
\end{itemize}

\noindent
\justify

Dispositivo dise\~nado para la extracci\'on de material contenido en cartuchos. Estos cartuchos constan de un cuerpo hueco de fondo abierto. En la parte superior de la carcasa se incluye un elemento de inyecci\'on con un punto que sobresale hacia el interior abierto que penetra el cartucho para el desarrollo del proceso de extracci\'on. 

\subsection{An\'alisis, comentarios y proyecciones}

\noindent
\justify

Los avances cient\'ificos expuestos comprenden el periodo de 1986 y 2020. La mayor\'ia de los estudios que involucran el m\'etodo MSPD est\'an dirigidos a mejorar la empleabilidad del m\'etodo a escala de laboratorio. A pesar de haber estudios realizados a dise\~nos escalables propuestos y de que existe una patente en tr\'amite que cimenta las bases de plantas de extracci\'on basadas en el m\'etodo MSPD, no existe una planta f\'isica que demuestre su viabilidad como proceso industrial. 

\noindent
\justify

En la base de datos de Scopus, s\'olo existe una revista especializada en la extracci\'on con solvente, titulada: \textit{``Solvent Extraction Research and Developement"}, la cual ha operado entre 1996 y 2020. Se han realizado publicaciones referentes al uso de m\'etodos extractivos en otras revistas, de las que se destacan: \textit{``TrAC - Trends in Analytical Chemistry"}, \textit{``Analytical Chemistry"}, \textit{``Food Chemistry"} y \textit{``Trends in Environmental Analytical Chemistry"}. Todo lo anterior, enfocado a desarrollo cient\'ifico a escala de laboratorio. 

\noindent
\justify

En el presente trabajo, se propone una nueva investigaci\'on sobre la escalabilidad del m\'etodo MSPD, \textbf{enfocado} en la etapa de eluci\'on y filtrado a trav\'es del desarrollo de un modelo CFD-DEM que permita la predicci\'on del comportamiento hidrodin\'amico entre el solvente a emplear y el material vegetal pulverizado.
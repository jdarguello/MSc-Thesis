\subsection{Actividades}

El desarrollo del proyecto se divide en \textbf{diez} etapas, cada una de ellas cuenta con diferentes tareas, o procesos, que permitir\'an alcanzar los objetivos propuestos.

\subsubsection{Planteamiento de la Investigaci\'on (PI)}

\noindent
\justify

Cumple el prop\'osito de conocer y comprender el objeto de estudio, permitiendo definir el planteamiento del problema, los objetivos del proyecto y cimentar las bases de la presente propuesta de investigaci\'on. Est\'a compuesta por las siguientes tareas:

\begin{itemize}
	\item \textit{Base investigativa (BI):} se identifica el problema a resolver y se define tanto el alcance del proyecto como los objetivos espec\'ificos a cumplir. Tiempo requerido: 2 semanas.
	\item \textit{Recopilaci\'on bibliogr\'afica (RB):} recolecci\'on de informaci\'on de inter\'es. Busca crear una fundamento s\'olido sobre el cual desarrollar la investigaci\'on. Tiempo requerido: 2 semanas.
	\item \textit{Selecci\'on de alternativas (SA):} a partir de los resultados obtenidos en la etapa de recopilaci\'on bibliogr\'afica, se plantean las alternativas de soluci\'on para el inicio del dise\~no conceptual. Tiempo requerido: 2 semanas.
\end{itemize}

\subsubsection{Dise\~no conceptual (DC)}

\noindent
\justify

Se plantea una metodolog\'ia de selecci\'on de la alternativa final de dise\~no, teniendo en cuenta par\'ametros como:

\begin{itemize}
	\item Costo de inversi\'on inicial.
	\item Consumo energ\'etico.
	\item Tiempo de eluci\'on.
\end{itemize}

\subsubsection{Dise\~no funcional (DF)}

\noindent
\justify

Conocido el sistema a dise\~nar, se procede a elaborar algoritmos de dise\~no empleando como referencia lo reportado en la literatura para definir caracter\'isticas operacionales y tama\~nos de referencia. 

\subsubsection{Modelo CFD}

\noindent
\justify

Desarrollo del modelo CFD para conocer el funcionamiento del sistema de eluci\'on y filtrado empleando los m\'etodos num\'ericos de vol\'umenes finitos (FVM) y elementos discretos (DEM).

\subsubsection{Pruebas experimentales (PE)}

\noindent
\justify

Pruebas que verifican la eficiencia y efectividad del dise\~no realizado basado en el modelo CFD desarrollado.

\subsubsection{Planos constructivos (PC)}

\noindent
\justify

Desarrollo de los planos constructivos del sistema de eluci\'on y filtrado para la planta de extracci\'on con capacidad de procesamiento de $20 \left[kg / bache \right]$, tres baches al d\'ia.

\subsubsection{Redacci\'on tesis (RT)}

\noindent
\justify

Espacio donde se espera poder redactar la tesis de maestr\'ia.
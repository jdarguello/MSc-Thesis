\paragraph{Trayectoria de una part\'icula en una lamela}

\noindent
\justify

En la pr\'actica, en sistemas de sedimentaci\'on de placas inclinadas, el \textit{estrangulamiento} que sufre el flujo en la entrada de la lamela incrementa la velocidad del fluido y, de manera impl\'icita, acelera las part\'iculas en esta misma direcci\'on. Esta aceleraci\'on se opone al peso de la part\'icula; impidiendo que la sedimentaci\'on se produzca sino hasta el punto en que la part\'icula s\'olida alcanza el equilibrio din\'amico con en el medio circundante gracias al comportamiento laminar del flujo. Es debido a ello que la Ecuaci\'on \ref{Uas} representa la velocidad de asentamiento m\'axima posible que podr\'ia alcanzar una part\'icula durante el proceso de sedimentaci\'on; siendo esta velocidad variable durante todo su recorrido dentro de la lamela.

\noindent
\justify

Es posible predecir una trayectoria \textit{aproximada} de una part\'icula con base en la simplificaci\'on adoptada en la secci\'on \ref{hidroD} y al an\'alisis cinem\'atico empleado en la secci\'on \ref{carga}. Conociendo el comportamiento de las velocidades en las componentes $x$ y $y$ (ver Figura \ref{vel_particula}), se obtiene la siguiente ecuaci\'on diferencial:

\begin{equation}
	v = \frac{dx}{dt}; \, u = \frac{dy}{dt}
	\label{EDs}
\end{equation}

\noindent
\justify

Combinando las Ecuaciones \ref{v}, \ref{u} y \ref{EDs}; se tiene:

\begin{equation}
	\frac{dy}{dx} = \frac{-U \cos \theta}{v(y)-U \sin \theta}
	\label{GeneralED}
\end{equation}

\noindent
\justify

Al integrar la Ecuaci\'on \ref{GeneralED}, se obtiene:

\begin{equation}
	\int v(y) dy - U \, y \sin \theta + U \, x \cos \theta = C_0
	\label{GeneralI}
\end{equation}

\noindent
\justify




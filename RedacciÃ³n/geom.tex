\subsection{Geometr\'ia} \label{geo}

\noindent
\justify

Para la definici\'on de la geometr\'ia, se desarroll\'o una metodolog\'ia de c\'alculo te\'orico autom\'atico en Python - Jupyter para diferentes problemas de inter\'es, conforme a la metodolog\'ia planteada en el Cap\'itulo \ref{teorico:sed}.

\noindent
\justify

El di\'ametro de la entrada de la mezcla \textit{solvente - material pulverizado} est\'a definido por la Ecuaci\'on \ref{D_I}.

\begin{equation}
	D_I = \frac{4 Q_T}{Re \, \pi \mu}
	\label{D_I}
\end{equation}

\noindent
\justify

En d\'onde: $D_I$ es el di\'ametro de ingreso de la mezcla s\'olido - l\'iquido, $Q_T$ es el caudal del problema, $Re$ es el n\'umero de Reynolds y $\mu$ es la viscosidad cinem\'atica del fluido.

\noindent
\justify

Con base en los resultados mostrados en el Cuadro \ref{resul_dis}, el di\'ametro de ingreso tiene un valor de $27 [cm]$. La geometr\'ia con la que se desarrolla el presente modelo se puede apreciar en la Figura \ref{geometria:CFD}.



\newpage

\noindent
\justify

De la Figura \ref{geometria:CFD}, el \'area sombreada corresponde a la geometr\'ia de an\'alisis, en donde se observa: una zona de \textit{entrada} de la mezcla s\'olido - l\'iquido; una de \textit{salida} (al final de la superficie inclinada), en donde se espera obtener la \textit{fase l\'iquida} de la mezcla; y un \'area de dep\'osito de lodos, localizada en la parte inferior de la geometr\'ia, en donde se deposita el material particulado.
\begin{center}
	\textbf{\textsc{{\Large Abstract}}}\\
\end{center}

\noindent
\justify

\begin{table}[h!]
\begin{tabular}{ll}
\textbf{T\'itulo:} & \noindent\parbox{0.6\textwidth}{Design of the elution and filtering system of an extraction plant\footnotemark} \\
 & \\
\textbf{Author:} & \noindent\parbox{0.6\textwidth}{Juan David Arg\"uello Plata\footnotemark} \\
 & \\
\textbf{Key words:} & \noindent\parbox{0.6\textwidth}{CFD-DEM, MSPD, Python, Jupyter, ParaView, OpenFoam, Yade} \\
\end{tabular}
\end{table}


\noindent
\justify


\textbf{\large Content:} 

\noindent
\justify

The extraction process based on the matrix solid-phase dispersion (MSPD) can be summarized in three steps. The first one is the pretreatment, which consists of decreasing the particle size of organic material to increase mass transfer area. The next one consists of the elution and filtering step, where the extraction of secondary metabolites is produced with the help of a solvent. Finally, a separation process is required to reuse the solvent for future extraction processes and to obtain the final product (extract).

\noindent
\justify


An automatic design methodology is proposed, from where the fluid dynamics behaviour is simulated during the filtering process, allowing to predict the particle concentration along the system through a numerical model based on CFD-DEM. This methodology had been elaborated with open source tools. Using Python as base language, Jupyter as development environment, ParaView as platform for analysis of results and C++ libraries (like Yade, LIGGGHTS and OpenFoam) for the developement of numerical simulations.


\footnotetext[3]{Master thesis project.}
\footnotetext[4]{\underline{Faculty:} Physical mechanical engineering. \underline{School:} Mechanical engineering. \underline{Director:} Omar Armando G\'elvez Arocha.}
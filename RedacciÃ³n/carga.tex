\subparagraph{Carga superficial} \label{carga}

\noindent
\justify

``Una part\'icula con velocidad de asentamiento $\vec{U}$, y transportada con velocidad $\vec{v}$, seguir\'ia una trayectoria rectil\'inea inclinada como resultado de la suma del vector de velocidad de flujo y del vector de velocidad de asentamiento, indicada por la recta $OB$"$^{\cite{RobertoRojas}}$.

\begin{figure}[h!]
	\centering
	\begin{tikzpicture}
		%Lamela
		\draw (0,0) -- ({\aancho*cos(90-\ang)},{\aancho*sin(90-\ang)});
		\draw[line width=1.0mm] (0,0) -- ({-\LL*cos(\ang)}, {\LL*sin(\ang)});
		\draw ({-\LL*cos(\ang)}, {\LL*sin(\ang)}) -- ({-\LL*cos(\ang)+\aancho*cos(90-\ang)}, {\LL*sin(\ang) + \aancho*sin(90-\ang)});
		\draw[line width=1.0mm] ({-\LL*cos(\ang)+\aancho*cos(90-\ang)}, {\LL*sin(\ang) + \aancho*sin(90-\ang)}) -- ({\aancho*cos(90-\ang)},{\aancho*sin(90-\ang)});
		
		%--Medidas--
		%b
		\draw ({-1*\LL*cos(\ang) + 0*\aancho*cos(90-\ang)-0.25*cos(\ang)}, {1*\LL*sin(\ang) + 0*\aancho*sin(90-\ang)+0.25*sin(\ang)}) -- ({-1*\LL*cos(\ang) + 0*\aancho*cos(90-\ang)-0.5*cos(\ang)}, {1*\LL*sin(\ang) + 0*\aancho*sin(90-\ang)+0.5*sin(\ang)});
		\draw ({-1*\LL*cos(\ang) + 1*\aancho*cos(90-\ang)-0.25*cos(\ang)}, {1*\LL*sin(\ang) + 1*\aancho*sin(90-\ang)+0.25*sin(\ang)}) -- ({-1*\LL*cos(\ang) + 1*\aancho*cos(90-\ang)-0.5*cos(\ang)}, {1*\LL*sin(\ang) + 1*\aancho*sin(90-\ang)+0.5*sin(\ang)});	
		
		\draw[-triangle 45] ({-1*\LL*cos(\ang) + 0*\aancho*cos(90-\ang)-0.375*cos(\ang)}, {1*\LL*sin(\ang) + 0*\aancho*sin(90-\ang)+0.375*sin(\ang)}) -- ({-1*\LL*cos(\ang) + 1*\aancho*cos(90-\ang)-0.375*cos(\ang)}, {1*\LL*sin(\ang) + 1*\aancho*sin(90-\ang)+0.375*sin(\ang)});
		
		\draw[-triangle 45] ({-1*\LL*cos(\ang) + 1*\aancho*cos(90-\ang)-0.375*cos(\ang)}, {1*\LL*sin(\ang) + 1*\aancho*sin(90-\ang)+0.375*sin(\ang)}) -- ({-1*\LL*cos(\ang) + 0*\aancho*cos(90-\ang)-0.375*cos(\ang)}, {1*\LL*sin(\ang) + 0*\aancho*sin(90-\ang)+0.375*sin(\ang)}) node [midway, above, rotate=\ang] {$b$};	
		
		%L
		\draw ({-0.25*cos(\ang)}, {-0.25*sin(\ang)}) -- ({-0.5*cos(\ang)}, {-0.5*sin(\ang)});
		
		\draw ({-1*\LL*cos(\ang)-0.25*cos(\ang)}, {1*\LL*sin(\ang)-0.25*sin(\ang)}) -- ({-1*\LL*cos(\ang)-0.5*cos(\ang)}, {1*\LL*sin(\ang)-0.5*sin(\ang)});
		
		\draw[-triangle 45, fill=black] ({-0.375*cos(\ang)}, {-0.375*sin(\ang)}) -- ({-1*\LL*cos(\ang)-0.375*cos(\ang)}, {1*\LL*sin(\ang)-0.375*sin(\ang)});
		
		\draw[-triangle 45, fill=black] ({-1*\LL*cos(\ang)-0.375*cos(\ang)}, {1*\LL*sin(\ang)-0.375*sin(\ang)}) -- ({-0.375*cos(\ang)}, {-0.375*sin(\ang)}) node [midway, below, rotate=-\ang] {$L$};
		
		%Theta
		\draw[dashed] (-0.75,0) -- (-0.3*\LL, 0);
		
		\draw (-0.15*\LL,0) arc (180:180-\ang:0.15*\LL) node [midway, left] {$\theta$};
		
		%Eje coordenado
		\draw[-triangle 45] (1, {0.8*\LL*sin(\ang)}) -- ({1+1.5*cos(\ang)}, {0.8*\LL*sin(\ang) + 1.5*sin(\ang)}) node [left=4mm, above, rotate=-\ang] {$y$};
		
		\draw[-triangle 45] (1, {0.8*\LL*sin(\ang)}) -- ({1-1.5*cos(\ang)}, {0.8*\LL*sin(\ang) + 1.5*sin(\ang)}) node [left=4mm, above, rotate=-\ang] {$x$};
		
		%g
		\draw[-triangle 45, dashed] (1, {0.8*\LL*sin(\ang)}) -- (1, 3) node [midway, left] {$\vec{g}$};
		
		%SEDIMENTOS
		\draw[pattern=dots, pattern color=brown!80!gray, dotted] (0,0) -- ({0.2*\aancho*cos(90-\ang)},{0.2*\aancho*sin(90-\ang)}) -- ({-0.85*\LL*cos(\ang)}, {0.85*\LL*sin(\ang)}) -- cycle;
		
		%--Análisis de la partícula--
		\draw[dotted] ({\aancho*cos(90-\ang)},{\aancho*sin(90-\ang)}) node [right=1mm] {$O$} -- ({-\LL*cos(\ang)}, {\LL*sin(\ang)}) node [left] {$B$};
		
		\draw[fill=brown!80!gray] ({-0.35*\LL*cos(\ang) + 0.65*\aancho*cos(90-\ang)}, {0.35*\LL*sin(\ang) + 0.65*\aancho*sin(90-\ang)}) circle (0.16);
		
		\draw[-triangle 45] ({-0.35*\LL*cos(\ang) + 0.65*\aancho*cos(90-\ang)}, {0.35*\LL*sin(\ang) + 0.65*\aancho*sin(90-\ang)}) -- ({-0.5*\LL*cos(\ang) + 0.65*\aancho*cos(90-\ang)}, {0.5*\LL*sin(\ang) + 0.65*\aancho*sin(90-\ang)}) node [left, above] {$\vec{v}$};
		
		\draw[-triangle 45] ({-0.35*\LL*cos(\ang) + 0.65*\aancho*cos(90-\ang)}, {0.35*\LL*sin(\ang) + 0.65*\aancho*sin(90-\ang)}) -- ({-0.35*\LL*cos(\ang) + 0.65*\aancho*cos(90-\ang)}, 2.2) node [below] {$\vec{U}$};
		
		\draw[dashed] plot [smooth] coordinates{({-0.35*\LL*cos(\ang) + 0.65*\aancho*cos(90-\ang)}, {0.35*\LL*sin(\ang) + 0.65*\aancho*sin(90-\ang)}) ({-0.55*\LL*cos(\ang) + 0.5*\aancho*cos(90-\ang)}, {0.55*\LL*sin(\ang) + 0.5*\aancho*sin(90-\ang)}) ({-0.8*\LL*cos(\ang)}, {0.8*\LL*sin(\ang)})};
		
		\draw[dashed, brown!80!gray, pattern=north west lines, pattern color=brown!80!gray] ({-0.55*\LL*cos(\ang) + 0.5*\aancho*cos(90-\ang)}, {0.55*\LL*sin(\ang) + 0.5*\aancho*sin(90-\ang)}) circle (0.16);
		
		\draw[dashed, brown!80!gray, pattern=north west lines, pattern color=brown!80!gray] ({-0.7*\LL*cos(\ang) + 0.22*\aancho*cos(90-\ang)}, {0.7*\LL*sin(\ang) + 0.22*\aancho*sin(90-\ang)}) circle (0.16);
		
		\draw[dashed, -triangle 45] ({-0.35*\LL*cos(\ang) + 0.65*\aancho*cos(90-\ang)}, {0.35*\LL*sin(\ang) + 0.65*\aancho*sin(90-\ang)}) -- ({-0.35*\LL*cos(\ang) + 0.35*\aancho*cos(90-\ang)}, {0.35*\LL*sin(\ang) + 0.35*\aancho*sin(90-\ang)}) node [rotate=-\ang, below=-1mm] {$u$};	
		
		%--Perfil de velocidad1--
		\draw[black!50!white] ({-0.81*\LL*cos(\ang)}, {0.81*\LL*sin(\ang)}) -- ({-0.81*\LL*cos(\ang) + 1*\aancho*cos(90-\ang)}, {0.81*\LL*sin(\ang) + 1*\aancho*sin(90-\ang)});
		
		\draw[black!50!white, name path global=curve] ({-0.81*\LL*cos(\ang)}, {0.81*\LL*sin(\ang)}) ..controls({-0.98*\LL*cos(\ang) + 0.5*\aancho*cos(90-\ang)}, {0.98*\LL*sin(\ang) + 0.5*\aancho*sin(90-\ang)}) .. ({-0.81*\LL*cos(\ang) + 1*\aancho*cos(90-\ang)}, {0.81*\LL*sin(\ang) + 1*\aancho*sin(90-\ang)});	
		
		\node[black!50!white, rotate=\ang] at ({-0.971*\LL*cos(\ang) + 0.5*\aancho*cos(90-\ang)}, {0.971*\LL*sin(\ang) + 0.5*\aancho*sin(90-\ang)}) {$\vec{v_0}$};
		
		\foreach \x in {1,...,7}
			\path[name path global/.expanded=vertical\x] ({-0.81*\LL*cos(\ang) + 0.125*\x*\aancho*cos(90-\ang)}, {0.81*\LL*sin(\ang) + 0.125*\x*\aancho*sin(90-\ang)}) -- ({-0.98*\LL*cos(\ang) + 0.125*\x*\aancho*cos(90-\ang)}, {0.98*\LL*sin(\ang) + 0.125*\x*\aancho*sin(90-\ang)});
		\foreach \x in {1,...,7}
			\draw[black!50!white, -triangle 45, name intersections={of=curve and {vertical\x}}] ({-0.81*\LL*cos(\ang) + 0.125*\x*\aancho*cos(90-\ang)}, {0.81*\LL*sin(\ang) + 0.125*\x*\aancho*sin(90-\ang)}) -- (intersection-1);		
		
		%--Perfil de velocidad2--
		\draw[gray, name path global=curve1] (0,0) ..controls({0.1*\LL*cos(\ang) + 0.1*\aancho*cos(90-\ang)}, {-0.1*\LL*sin(\ang) + 0.1*\aancho*sin(90-\ang)}) .. ({-0*\LL*cos(\ang) + 0.2*\aancho*cos(90-\ang)}, {0*\LL*sin(\ang) + 0.2*\aancho*sin(90-\ang)}) .. controls({-0.17*\LL*cos(\ang) + 0.6*\aancho*cos(90-\ang)}, {0.17*\LL*sin(\ang) + 0.6*\aancho*sin(90-\ang)}) .. ({0*\LL*cos(\ang) + 1*\aancho*cos(90-\ang)}, {0*\LL*sin(\ang) + 1*\aancho*sin(90-\ang)});
		
		\foreach \x in {1,...,4}
			\path[name path global/.expanded=ver\x] ({0*\LL*cos(\ang) + 0.05*\x*\aancho*cos(90-\ang)}, {-0*\LL*sin(\ang) + 0.05*\x*\aancho*sin(90-\ang)}) -- ({0.1*\LL*cos(\ang) + 0.05*\x*\aancho*cos(90-\ang)}, {-0.1*\LL*sin(\ang) + 0.05*\x*\aancho*sin(90-\ang)});
			
		\foreach \x in {1, ..., 3}
			\draw[gray, -triangle 45, name intersections={of=curve1 and {ver\x}}] ({0*\LL*cos(\ang) + 0.05*\x*\aancho*cos(90-\ang)}, {-0*\LL*sin(\ang) + 0.05*\x*\aancho*sin(90-\ang)}) -- (intersection-1);
			
		\foreach \x in {1, ..., 5}
			\path[name path global/.expanded=ver\x] ({-0*\LL*cos(\ang) + (0.2+0.133*\x)*\aancho*cos(90-\ang)}, {0*\LL*sin(\ang) + (0.2+0.133*\x)*\aancho*sin(90-\ang)}) -- ({-0.17*\LL*cos(\ang) + (0.2+0.133*\x)*\aancho*cos(90-\ang)}, {0.17*\LL*sin(\ang) + (0.2+0.133*\x)*\aancho*sin(90-\ang)});
			
		\foreach \x in {1, ..., 5}
			\draw[gray, -triangle 45, name intersections={of=curve1 and {ver\x}}] ({-0*\LL*cos(\ang) + (0.2+0.133*\x)*\aancho*cos(90-\ang)}, {0*\LL*sin(\ang) + (0.2+0.133*\x)*\aancho*sin(90-\ang)}) -- (intersection-1);
		
		\node[gray, below=-1.5mm, rotate=\ang] at ({0.1*\LL*cos(\ang) + 0.1*\aancho*cos(90-\ang)}, {-0.1*\LL*sin(\ang) + 0.1*\aancho*sin(90-\ang)}) {$\vec{v_s}$};
		
		\node[gray, rotate=\ang] at ({-0.17*\LL*cos(\ang) + 0.6*\aancho*cos(90-\ang)}, {0.17*\LL*sin(\ang) + 0.6*\aancho*sin(90-\ang)}) {$\vec{v_0}$};
		
		
			
	\end{tikzpicture}
	\caption{Cinem\'atica de una part\'icula s\'olida.}
	\label{vel_particula}
\end{figure}

\noindent
\justify

De la Figura \ref{vel_particula}, $\vec{v}$ corresponde a la velocidad horizontal (a favor del flujo), su valor depende de la posici\'on $y$ en la que se encuentre. $\vec{v_s}$ se refiere a la velocidad de salida de los sedimentos. Para el presente an\'alisis, se asume que la part\'icula se encuentra en el punto de velocidad m\'aximo $\vec{v_0}$; $\vec{U}$ corresponde a la velocidad de asentamiento de la part\'icula y $u$ es la componente en la direcci\'on $y^{-}$ de la velocidad de asentamiento.

\noindent
\justify

Por semejanza de tri\'angulos, se obtiene la siguiente relaci\'on matem\'atica:

\begin{equation}
	\frac{v}{u} = \frac{L}{b}
	\label{triS}
\end{equation}

\noindent
\justify

Debido a la inclinaci\'on, existe una componente de la velocidad de asentamiento que se opone al movimiento de la part\'icula sobre la lamela; de modo que:

\begin{equation}
	v = v_0 - U \sin \theta = \frac{Q _l}{b \, W} - U \sin \theta
	\label{v}
\end{equation}

\noindent
\justify

D\'onde $W$ es el ancho del tanque de sedimentaci\'on y $Q_l$ el caudal dentro de la lamela. De igual forma, el valor de $u$ corresponde a la magnitud de $\vec{U}$ en la componente $y^{-}$.

\begin{equation}
	u = U \cos \theta
	\label{u}
\end{equation}

\noindent
\justify

Relacionando las Ecuaciones \ref{triS}, \ref{v} y \ref{u} se obtiene:

\begin{equation}
	\boxed{U = \frac{Q_l}{\left(\frac{L}{b} +  \tan \theta \right) b W \cos \theta } }
	\label{cargaS}
\end{equation}

\noindent
\justify

Si el \'angulo de inclinaci\'on es $0 [\degree]$, la Ecuaci\'on \ref{cargaS} se reduce a lo siguiente:

\begin{equation}
	U = \frac{Q_l}{L W} = \frac{Q_l}{A} = C_s
	\label{Crgs}
\end{equation}

\noindent
\justify

La Ecuaci\'on \ref{Crgs} se conoce en la literatura$^{\cite{RobertoRojas, PerezParra1997}}$ como \textit{carga superficial} $(C_s)$; la cual define la sedimentaci\'on como una funci\'on del \'area superficial de las lamelas. ``Todas las part\'iculas discretas con velocidad de asentamiento igual o mayor que $U$ ser\'an completamente removidas"$^{\cite{RobertoRojas}}$.

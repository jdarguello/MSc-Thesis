\begin{center}
	\section{Alcance y resultados}
\end{center}

\noindent
\justify

Conforme a los objetivos planteados en la secci\'on anterior, se propone la siguiente metodolog\'ia de trabajo por objetivo.

\noindent
\justify

\subsection{Propiedades termodin\'amicas del solvente}

\noindent
\justify

A partir de metodolog\'ias de c\'alculo de las propiedades termodin\'amicas (presi\'on, temperatura, densidad, calor espec\'ifico y entalp\'ia, entre otras) de mezclas reportadas en la literatura, se plantear\'a un algoritmo, hecho en Python - Jupyter, que permita describir el estado termodin\'amico de mezclas hidroalcoh\'olicas cambiando par\'ametros como: concentraci\'on de la mezcla, temperatura y presi\'on.

\subsection{Construcci\'on del modelo CFD}

\noindent
\justify

Una vez desarrollado el algoritmo de propiedades termodin\'amicas, se desarrollar\'ia el modelo CFD (\textit{``Computational Fluid Dynamics"}) en Python - Jupyter, empleando las librer\'ias OpenFOAM y Yade para la simulaci\'on fluidodin\'amica entre el solvente y el material vegetal pulverizado. El algoritmo desarrollado se elaborar\'a de tal forma que permita la simulaci\'on autom\'atica del sistema durante la etapa de filtrado, seleccionado previamente en el \textit{Dise\~no conceptual}. Dicho algoritmo permitir\'a, a su vez, elaborar un informe de ingenier\'ia autom\'atico sobre los diferentes resultados obtenidos de la simulaci\'on.

\noindent
\justify

El cumplimiento de este objetivo tendr\'a como resultados:

\begin{itemize}
	\item La publicaci\'on de los algoritmos desarrollados en una revista especializada de dise\~no mec\'anico y/o de software libre.
	\item Publicaci\'on de la metodolog\'ia de la simulaci\'on en un congreso internacional en la modalidad de ponencia oral o p\'oster.
\end{itemize}

\subsection{Pruebas experimentales}

\noindent
\justify

Desarrollo de un estudio experimental que permita comparar y validar los resultados obtenidos del modelo CFD, tomando muestras de la mezcla en tres puntos clave del proceso (punto inicial, medio y final), para la evaluaci\'on de la concentraci\'on de la misma. En estos mismos puntos, se evaluar\'a tambi\'en valores como temperatura y velocidad de flujo, que se emplear\'an tambi\'en como referencia. Como resultado, se esperar\'ia poder realizar una publicaci\'on en un congreso internacional, en la modalidad de ponencia oral o p\'oster, exponiendo los resultados de la prueba comparativa. 

\subsection{Planos constructivos}

\noindent
\justify

Una vez validado el modelo CFD, se procede al desarrollo de los planos constructivos respectivos del sistema de eluci\'on y filtrado que se adapte a la capacidad objetivo de la planta de extracci\'on de $20 \left[kg / bache \right]$, tres baches al d\'ia.

\begin{center}
	\section{Objetivos}
\end{center}

\subsection{Objetivo general}

\begin{itemize}
	\item Desarrollar una metodolog\'ia para el dise\~no del sistema de eluci\'on y filtrado, de la planta de extracci\'on, que permita determinar la concentraci\'on de part\'iculas\footnote{Corresponde al material pulverizado proveniente de la etapa de molienda (ver Figura \ref{cadena}).} durante la etapa de filtrado a trav\'es de un modelo num\'erico, validado mediante experimentaci\'on.
\end{itemize}

\subsection{Objetivos espec\'ificos}

\begin{itemize}
	\item Construir el modelo CFD-DEM que simule el comportamiento fluidodin\'amico durante la etapa de filtrado, para una geometr\'ia dada del sistema, donde se obtengan los siguientes resultados: perfiles de velocidad, de presi\'on y grado de concentraci\'on de part\'iculas a lo largo del sistema virtual en funci\'on de la relaci\'on m\'asica entre el material pulverizado y el solvente (mezcla etanol - agua al $50\%$).
	\item Implementar el modelo CFD-DEM a trav\'es de herramientas de c\'odigo abierto; utilizando Python como lenguaje base, Jupyter como entorno de desarrollo, ParaView como plataforma de an\'alisis de resultados y librer\'ias de C++ (como Yade, LIGGGHTS y OpenFoam) para el desarrollo de las simulaciones num\'ericas.
	\item Evaluar el modelo desarrollado mediante pruebas experimentales realizadas a un sistema f\'isico real que emplee el mismo principio f\'isico-qu\'imico de separaci\'on de sustancias, tomando como variable de entrada la relaci\'on m\'asica entre el material pulverizado y el solvente (mezcla etanol - agua al $50\%$), y evaluando como variable de salida la concentraci\'on de part\'iculas en los puntos inicial, medio y final del sistema.
\end{itemize}
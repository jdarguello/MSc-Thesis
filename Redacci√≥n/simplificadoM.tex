\subparagraph{Dise\~no anal\'itico}

\noindent
\justify

Hasta el momento, el an\'alisis te\'orico se ha enfocado en desarrollar un an\'alisis aproximado de la interacci\'on fluido-part\'icula en \underline{una} lamela. En la presente secci\'on, se complementa esta informaci\'on con una metodolog\'ia de dimensionamiento del sistema de sedimentaci\'on.

\noindent
\justify

Como se aprecia en la Figura \ref{lamelas_inclinadas}, el sedimentador de placas paralelas subdivide el espacio en compartimientos (lamelas). Esta configuraci\'on geom\'etrica cumple dos objetivos: incrementa el \'area de sedimentaci\'on y origina el flujo laminar$^{\cite{Lekang2001}}$. La inclinaci\'on de las placas permite el deslizamiento de los sedimentos gracias a la gravedad y a la diferencia de densidades respecto al fluido circundante (ver Figura \ref{teo_sed}). El sedimentador debe contar con una tolva c\'onica en el fondo del recipiente para la recolecci\'on y expulsi\'on de lodos. Es importante destacar que durante toda la operaci\'on, el flujo debe ser laminar; de modo que se recomienda que el n\'umero de Reynolds sea inferior a 500$^{\cite{Robescu2010}}$. 

\noindent
\justify

Los par\'ametros m\'as empleados en el dise\~no de sedimentadores son la \textit{carga superficial} y el \textit{\'area superficial}$^{\cite{RobertoRojas}}$. En el Cuadro \ref{critSed}, se muestran los criterios de dise\~no sugeridos por Romero$^{\cite{RobertoRojas}}$ y P\'erez$^{\cite{PerezParra1997}}$.

\begin{table}[h!]
	\centering
	\begin{tabular}{|c|c|}
		\hline
		\textbf{Par\'ametro} & \textbf{Valor} \\ \hline
		Carga superficial $C_s$ & $6.0 - 180 [m/d]$ \\ \hline
		Tiempo de retenci\'on en placas $t_p$ & $8 - 25 [min]$ \\ \hline
		Inclinaci\'on de placas $\theta$ & $60 [\degree]$ \\ \hline
		N\'umero de Reynolds $Re$ & $\le 500$ \\ \hline
		Distancia entre placas & $5 [cm]$ \\ \hline
		Velocidad cr\'itica de asentamiento $V_{sc}$ & $15 - 60 [m/d]$ \\ \hline
		Relaci\'on longitud-distanciamiento entre placas $l/d$ & $\ge 8$ \\ \hline
	\end{tabular}
	\caption{Criterios de dise\~no.}
	\label{critSed}
\end{table}

\noindent
\justify

Para el desarrollo del dise\~no, se debe conocer el caudal de entrada, la relaci\'on m\'asica entre el material s\'olido con el fluido y las propiedades del fluido. El tiempo de retenci\'on en las lamelas se calcula de acuerdo a lo expuesto en la Ecuaci\'on \ref{tp}.

\begin{equation}
	t_p = \frac{L}{v_0}
	\label{tp}
\end{equation}

\noindent
\justify

D\'onde: $l$ es el largo de las placas y $v_0$ es la velocidad promedio del fluido en la lamela. La carga superficial es equivalente a:

\noindent
\justify

Es posible adaptar la Ecuaci\'on \ref{cargaS} para relacionar la carga superficial con todas las dimensiones del panel de lamelas:

\begin{equation}
	U = \frac{Q}{N \left(\frac{L}{b} + \tan \theta \right) b W \cos \theta}
	\label{carS}
\end{equation}

\noindent
\justify

De la Ecuaci\'on \ref{carS}, $U$ se refiere a la carga superficial, $Q$ al caudal total de entrada al sistema de sedimentaci\'on, $N$ es el n\'umero de lamelas, $L$ es el largo cada lamela, $b$ es el ancho de una lamela, $W$ es la profundidad del panel de lamelas y $\theta$ es el \'angulo de inclinaci\'on del panel.

\noindent
\justify

La \textbf{metodolog\'ia de dise\~no} consiste en desarrollar un algoritmo iterativo tal que se minimice la carga superficial, de acuerdo a lo recomendado en el Cuadro \ref{critSed}, con las menores dimensiones posibles (a menor tama\~no de panel, menor costo de inversi\'on) y garantizando un valor de Reynolds en r\'egimen laminar, menor a $500$, en cada una de las lamelas.

\noindent
\justify

El n\'umero de Reynolds de una lamela se calcula conforme a la Ecuaci\'on \ref{Re}$^{\cite{streeter1951fluid}}$.

\begin{equation}
	Re = \frac{v_0 \, b}{\mu}
	\label{Reynolds}
\end{equation}

\noindent
\justify

De la Ecuaci\'on \ref{Reynolds}: $\mu$ se refiere a la viscosidad cinem\'atica del fluido, $b$ es la separac\'on entre placas y $v_0$ es la velocidad promedio del fluido en la lamela. La velocidad promedio corresponde a lo siguiente:

\begin{equation}
	v_0 = \frac{Q}{N \, b W}
	\label{v0}
\end{equation}

% TOMAR LA ECUACIÓN DE CARGA SUPERFICIAL Y EXPLICAR LA METODOLOGÍA DE DISEÑO ENTRE VELOCIDAD, CAUDAL, NÚMERO DE LAMELAS Y DIMENSIONES DE LA LAMELA. TENER EN CUENTA EL REYNOLDS COMO PARÁMETRO DE SELECCIÓN DE GEOMETRÍA EN LA LAMELA. LUEGO, TOMAR LA SEDIMENTACIÓN CRÍTICA Y DEFINIRLA COMO UN PARÁMETRO DE EFICIENCIA. LA LONGITUD CORREGIDA TOMARLA DEL PAPER BUENO QUE VI.

\noindent
\justify

La eficiencia del sistema de sedimentaci\'on se estima como la relaci\'on entre la velocidad cr\'itica de sedimentaci\'on (Ecuaci\'on \ref{Uc}) y la velocidad de asentamiento m\'axima (Ecuaci\'on \ref{Uas}).


\noindent
\justify

Romero$^{\cite{RobertoRojas}}$ estipula que el n\'umero de placas del sistema de sedimentaci\'on se calcula con base en la siguiente expresi\'on:

\begin{equation}
	N = \frac{L_s \sin \theta + d}{d+e}
	\label{NL}
\end{equation}

\noindent
\justify

D\'onde: $L_s$ es el longitud de la lamela, $d$ es la separaci\'on entre placas, $e$ es el espesor de las placas y $\theta$ es el \'angulo de inclinaci\'on.

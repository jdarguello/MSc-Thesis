\begin{center}
	\section{Conclusiones}
\end{center}


\begin{enumerate}
	\item Se desarroll\'o una metodolog\'ia de dise\~no num\'erico de un sistema de eluci\'on y filtrado para una planta de extracci\'on, en d\'onde se apreci\'o la interacci\'on fluido - part\'icula durante la etapa de filtrado, basada en el fen\'omeno de sedimentaci\'on. El modelo num\'erico fue validado, llegando a reportar m\'argenes de precisi\'on en los perfiles de velocidad de flujo inferiores al $1 \%$.
	\item A las condiciones de operaci\'on dadas ($0.3 \left[ m^3 \right]$ de caudal, $27 \left[ \degree C \right]$ de temperatura de operaci\'on y $250 [ \mu m]$ de tama\~no de part\'icula medio) las simulaciones num\'ericas del sistema de eluci\'on y filtrado demostraron un nivel de eficiencia general superior al $90 \%$ en la remoci\'on de material particulado; hecho que va acorde a los resultados reportados en el dise\~no global (cap\'itulo \ref{teorico:sed}).
	\item Con base en los resultados obtenidos, se puede apreciar la importancia de las simulaciones num\'ericas basadas en CFD-DEM para el dise\~no de maquinaria y equipos que busquen separar mezclas de sustancias s\'olidas y l\'iquidas. De la Figura \ref{CFDEM:part} se concluye que el panel de lamelas presenta zonas de salidas \textit{limpias} de material particulado y otras con una concentraci\'on menor a la inicialmente procesada. Futuras investigaciones podr\'ian emplear el modelo CFD-DEM desarrollado en el presente trabajo como base para el dise\~no de un nuevo sistema de sedimentaci\'on que emplee un \textit{reflujo} de las zonas con posibles impurezas para garantizar una completa separaci\'on de sustancias.
	\item Se implement\'o la metodolog\'ia del modelo CFD-DEM con base en herramientas de c\'odigo abierto; demostrando la viabilidad de estas herramientas en el desarrollo de nuevos productos y servicios de bajos costos de inversi\'on inicial y alto grado de innovaci\'on.
	\item De acuerdo al proceso de validaci\'on desarrollado en el cap\'itulo \ref{validacion}, se concluye que el modelo CFD-DEM desarrollado es preciso (error inferior al $1\%$ con respecto a los resultados experimentales) y puede ser empleado en diversas aplicaciones de flujo, bien sea laminar o turbulento, en donde part\'iculas s\'olidas interact\'uan con fluidos newtonianos en fase l\'iquida o gaseosa.
\end{enumerate}

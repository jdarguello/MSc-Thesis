\subsection{Geometr\'ia} \label{geo}

\noindent
\justify

Para la definici\'on de la geometr\'ia, se desarroll\'o una metodolog\'ia de c\'alculo te\'orico autom\'atico en Python - Jupyter para diferentes problemas de inter\'es, conforme a la metodolog\'ia planteada en el Cap\'itulo \ref{teorico:sed}.

\noindent
\justify

El di\'ametro de la entrada de la mezcla \textit{solvente - material pulverizado} est\'a definido por la Ecuaci\'on \ref{D_I}.

\begin{equation}
	D_I = \frac{4 Q_T}{Re \, \pi \mu}
	\label{D_I}
\end{equation}

\noindent
\justify

En d\'onde: $D_I$ es el di\'ametro de ingreso de la mezcla s\'olido - l\'iquido, $Q_T$ es el caudal del problema, $Re$ es el n\'umero de Reynolds y $\mu$ es la viscosidad cinem\'atica del fluido.

\noindent
\justify

Con base en los resultados mostrados en el Cuadro \ref{resul_dis}, el di\'ametro de ingreso tiene un valor de $27 [cm]$. La geometr\'ia con la que se desarrolla el presente modelo se puede apreciar en la Figura \ref{geometria:CFD}.

%Datos geométricos
\def\D{2.7}    	%Diámetro de ingreso
\def\H{3}		%Altura de la zona de lodos
\def\A{2.5}		%Ancho del panel de lamelas
\def\HL{3}		%Altura del panel de lamelas
\def\nL{5}		%Número de lamelas
\def\ang{60}	%Ángulo de inclinación del panel

\def\xlam{\x+\A+0.5*\D}

\def\fin{\nL-1}

%Inclinación
%\def\xL{{\HL*cos(\ang)}}

%Referencia
\def\x{-\A/2-0.5/2*\D-\HL/2}
\def\y{0}

\begin{figure}[h!]
	\centering
	\begin{adjustbox}{max width = \textwidth}
	\begin{tikzpicture}
		%Geometría general
		\draw (\x,\y) -- (\x,-\D) -- (\x+0.5*\D,-\D) --
			(\x+0.5*\D, -2*\D) -- ({\x+0.5*\D + \A/3}, -2.5*\D) -- ({\x+0.5*\D + 2*\A/3}, -2.5*\D) -- (\x+\A+0.5*\D, -2*\D) -- 
			(\x+\A+0.5*\D, \y) -- ({\xlam + \HL*cos(\ang)},{\y + \HL*sin(\ang)}) -- ({\xlam + \HL*cos(\ang) - \A}, {\y + \HL*sin(\ang)}) -- (\xlam - \A, \y) -- cycle;
						
		%Lamelas
		\foreach \xL in {1, ..., \nL}
			\draw (\xlam - \xL*\A/\nL,\y) -- ({\xlam + \HL*cos(\ang) - \xL*\A/\nL}, {\y + \HL*sin(\ang)});
		\foreach \xL in {1, ..., \nL}	
			\draw[dashed, step=0.5cm, blue!60!green, pattern=north west lines, pattern color=blue!60!green] ({\xlam - (\xL-1)*\A/\nL},\y) -- (\xlam - \xL*\A/\nL,\y) -- ({\xlam + \HL*cos(\ang) - \xL*\A/\nL}, {\y + \HL*sin(\ang)}) -- ({\xlam + \HL*cos(\ang) - (\xL-1)*\A/\nL}, {\y + \HL*sin(\ang)})  -- cycle;
			
		%Dimensiones
		\draw (\x+0.5*\D + 0.1, -2*\D) -- (\x, -2*\D);
		\draw[arrows={-Triangle[angle=90:3pt,red!10!black,fill=red!10!black]}] (\x+0.25*\D, -2*\D) -- (\x+0.25*\D, -\D);
		\draw[arrows={-Triangle[angle=90:3pt,red!10!black,fill=red!10!black]}] (\x+0.25*\D, -\D)  -- (\x+0.25*\D, -2*\D);
		\node[align=left] at (\x-0.1, -3/2*\D) {$29.4 [cm]$};
		
		\draw (\x+0.5*\D + 0.1 + \A, -2*\D) -- (\x+\D + 0.1 + \A, -2*\D);
		\draw (\x+0.5*\D + 0.1 + \A, \y) -- (\x+1.5*\D + 0.1 + \A, \y);
		
		\draw[arrows={-Triangle[angle=90:3pt,red!10!black,fill=red!10!black]}] (\x+0.75*\D + \A, -2*\D) -- (\x+0.75*\D+ \A, \y);
		\draw[arrows={-Triangle[angle=90:3pt,red!10!black,fill=red!10!black]}] (\x+0.75*\D+ \A, \y) --  (\x+0.75*\D+ \A, -2*\D);
		
		\node[align=right] at (\x+1.05*\D+ \A, -\D) {$56.4 [cm]$};
		
		\draw ({\xlam + \HL*cos(\ang)},{\y + \HL*sin(\ang)}) -- ({\xlam + \HL*cos(\ang) + \D},{\y + \HL*sin(\ang)});

		\draw (\x+0.5*\D, -2*\D - 0.1) -- (\x+0.5*\D, -2.8*\D - 0.1);
		\draw (\x+0.5*\D + \A, -2*\D - 0.1) -- (\x+0.5*\D + \A, -2.8*\D - 0.1);	
		\draw[arrows={-Triangle[angle=90:3pt,red!10!black,fill=red!10!black]}] (\x+0.5*\D + \A, -2.6*\D-0.1) -- (\x+0.5*\D, -2.6*\D - 0.1);
		\draw[arrows={-Triangle[angle=90:3pt,red!10!black,fill=red!10!black]}] (\x+0.5*\D, -2.6*\D - 0.1) -- (\x+0.5*\D + \A, -2.6*\D-0.1);
		
		\node at (\x+0.5*\D + \A/2, -2.7*\D - 0.1) {$35 [cm]$};	
		
		\draw[arrows={-Triangle[angle=90:3pt,red!10!black,fill=red!10!black]}] ({\xlam + \HL*cos(\ang) + \D/4}, {\y + \HL*sin(\ang)}) -- ({\xlam + \HL*cos(\ang) + \D/4}, \y);
		
		\draw[arrows={-Triangle[angle=90:3pt,red!10!black,fill=red!10!black]}] ({\xlam + \HL*cos(\ang) + \D/4}, \y) --  ({\xlam + \HL*cos(\ang) + \D/4}, {\y + \HL*sin(\ang)});
		
		\node at ({\xlam + \HL*cos(\ang) + \D/4 + 0.75}, {\y + \HL*sin(\ang)/2}) {$34.6 [cm]$};
		
		%Área de interés
		\draw[dashed, step=0.5cm, blue!60!green, pattern=north west lines, pattern color=blue!60!green] (\x,\y) -- (\x,-\D) -- (\x+0.5*\D,-\D) -- (\x+0.5*\D, -2*\D) -- (\x+\A+0.5*\D, -2*\D) -- (\x+\A+0.5*\D, \y) -- cycle;
		
		%Señalización
		\draw[red!10!black, arrows={-Triangle[angle=90:3pt,red!10!black,fill=red!10!black]}] (\x-2,-\D/2) node[above] {Entrada \textit{mezcla}} -- (\x-0.5,-\D/2);
		
		\draw[red!10!black, arrows={-Triangle[angle=90:3pt,red!10!black,fill=red!10!black]}] ({\xlam + \HL*cos(\ang) - \A/\nL/2},{\y + \HL*sin(\ang)}) node[above] {Salida \textit{fase l\'iquida}} -- ({\xlam + (\HL+1)*cos(\ang) - \A/\nL/2},{\y + (\HL+1)*sin(\ang)});
		
		
	\end{tikzpicture}
	\end{adjustbox}
	\caption{Vista en corte de la geometr\'ia del sistema de sedimentaci\'on.}
	\label{geometry}
\end{figure}

\newpage

\noindent
\justify

De la Figura \ref{geometria:CFD}, el \'area sombreada corresponde a la geometr\'ia de an\'alisis, en donde se observa: una zona de \textit{entrada} de la mezcla s\'olido - l\'iquido; una de \textit{salida} (al final de la superficie inclinada), en donde se espera obtener la \textit{fase l\'iquida} de la mezcla; y un \'area de dep\'osito de lodos, localizada en la parte inferior de la geometr\'ia, en donde se deposita el material particulado.